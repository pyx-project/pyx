% $Header$
%
\documentclass{article}
\usepackage{pyx}
\begin{document}
\section*{Design rules about global variable dependencies}

I've just noticed a problem in evaluating default arguments of
functions in combination with global variables. Consider the following
minimal example:

\begin{verbatim}
scale = 1

def a():
    print scale

def b(a = a()):
    pass

scale = 2
b()
\end{verbatim}

The question arises, what the output of this program will be. I
expected the output to be \verb|2|, but I observed \verb|1|. It means
that the modification of the scale doesn't effect the default argument
evaluation. This effect could easily confuse users of \PyX. We should
therefore avoid constructions where those early evaluations of default
arguments are influenced by global variables. Typical candidates are
constructors which will convert \PyX-length using the global unit
settings. When those classes are used as default arguments of
functions the effect shown above will occure. We should resolve this
untimely evaluation by moving \PyX-length interpretations out of the
constructors.

\end{document}
