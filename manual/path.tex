\chapter{Module path: PostScript like paths}

\label{path}

With help of the path module, one is able to construct PostScript like 
paths, which are one of the main building blocks for the generation of 
drawings. To that end it provides 
\begin{itemize}
\item classes (derived from \verb|pathel|) for the primitives \verb|moveto|, \verb|lineto|, etc.
\item the class \verb|path| (and derivatives thereof) representing an entire PostScript path
\end{itemize}

\section{Class pathel}

The class \verb|pathel| is the superclass of all PostScript path
construction primitives. It is never used directly, but only by
instantiating its subclasses, which correspond one by one to the
PostScript primitives.

\medskip
\begin{tabularx}{\linewidth}{>{\hsize=.7\hsize}X>{\raggedright\arraybackslash\hsize=1.3\hsize}X}
Subclass of \texttt{pathel} & Function \\
\hline
\texttt{closepath()} & closes current subpath \\
\texttt{moveto(x, y)} & sets current point to (\texttt{x},
\texttt{y})\\
\texttt{rmoveto(dx, dy)} & moves current point relative by (\texttt{dx},
\texttt{dy})\\
\texttt{lineto(x, y)} & appends straight line from current point to
(\texttt{x}, \texttt{y})\\
\texttt{rlineto(dx, dy)} & appends straight line from current point
relative by (\texttt{dx}, \texttt{dy})\\
\texttt{arc(x, y, r, \newline\phantom{arc(}angle1, angle2)} & appends arc segment in
counterclockwise direction with center (\verb|x|, \verb|y|) and
radius~\verb|r| from \verb|angle1| to \verb|angle2| (in degrees).\\
\texttt{arcn(x, y, r, \newline\phantom{arcn(}angle1, angle2)} & appends arc segment in
clockwise direction with center (\verb|x|, \verb|y|) and
radius~\verb|r| from \verb|angle1| to \verb|angle2| (in degrees). \\
\texttt{arct(x1, y1, x2, y2, r)} & appends arc segment with radius \verb|r|
which connects between (\verb|x1|, \verb|y1|) and (\verb|x2|,
\verb|y2|).\\
\texttt{rcurveto(dx1, dy1, \newline\phantom{rcurveto(}dx2, dy2,\newline\phantom{rcurveto(}dx3, dy3)} & appends a B\'ezier curve with
the control points current point, and the points defined relative to
the current point by (\verb|dx1|, \verb|dy1|), 
(\verb|dx2|, \verb|dy2|), and (\verb|dx3|, \verb|dy3|)
\end{tabularx}
\medskip

Some notes on the above:
\begin{itemize}
\item All coordinates are in \PyX\ lengths
\item If the current point is defined before an \verb|arc| or
  \verb|arcn| command, a straight line from current point to the
  beginning of the arc is prepended.
\item The bounding box (see below) of B\'ezier curves is actually only
  the control box, \textit{i.e.}\ not neccesarily the smallest
  enclosing rectangle.
\end{itemize}


\section{Class path}

The class path represents PostScript like paths in \PyX. The \verb|path| constructor allows the 
creation of such a path out of series of \verb|pathel|s. A simple example, which generates a triangle,
looks like:
\begin{quote}
\begin{verbatim}
import pyx
from path import *

p = path(moveto(0, 0), 
         lineto(0, 1),
         lineto(1, 1),
         closepath())
\end{verbatim}
\end{quote}
Later on, we shall see, how it is possible to output such a path on a
canvas. For the moment, we only want to discuss the methods provided
by the \verb|path| class. This range from standard operation like the
determination of the length of a path via \verb|len(p)|, fetching of
items using \verb|p[index]| and the possibility to concatenate two
paths, \verb|p1 + p2|, append further \verb|pathel|s using
\verb|p.append(pathel)| to more advanced methods, we shall discuss
now.  The first one is \verb|bbox()|, which returns the bounding box
of the path, \textit{i.e.} the smallest rectangle enclosing the path,
as an instance of the class \verb|bbox.bbox|. This is mainly used
during the output to an EPS file, but it can also be useful for your
own calculations. The second one, \verb|bpath()| converts the path to
a path consiting of B\'ezier curves and returns an instance of the class
\verb|bpath.bpath| (cf.\ Sect.~\ref{bpath}), Use this, if you want to
apply the more sophisticated methods, that are only available for
B\'ezier paths, like transformations, intersection, dissections, etc.


\section{Subclasses of path}

For your convenience, some special PostScript paths are already defined, which
are given in the following table.

\medskip
\begin{tabularx}{\linewidth}{l>{\raggedright\arraybackslash}X}
Subclass of \texttt{path} & Function \\
\hline
\texttt{line(x1, y1, x2, y2)} & a line from the point
  (\texttt{x1}, \texttt{y1} to the point (\texttt{x2}, \texttt{y2})\\
\texttt{rect(x, y, w, h)} &  a rectangle with the
  lower left point (\texttt{x}, \texttt{y}), width~\texttt{w}, and
  height~\texttt{h}. \\
\texttt{circle(x, y, r)} & a circle with 
  center (\texttt{x}, \texttt{y}) and radius~\texttt{r}.
\end{tabularx}
\medskip


%%% Local Variables:
%%% mode: latex
%%% TeX-master: "manual.tex"
%%% End:
