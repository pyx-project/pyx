\chapter{Module path: PostScript like paths}

\label{path}

\section{Introduction}

The path module allows one to construct PostScript-like
\textit{paths}, which are one of the main building blocks for the
generation of drawings. A PostScript path is an arbitrary shape built
up of straight lines, arc segments and cubic Bezier curves. Such a
path does not have to be connected but may also consist of multiple
connected segments, which will be called \textit{sub paths} in the
following.

Usually, a path is constructed by passing a list of the path
primitives \verb|moveto|, \verb|lineto|, \verb|curveto|, etc., to the
constructor of the \verb|path| class. The following code snippet, for
instance, defines a path \verb|p| that consists of a straight line
from the point $(0, 0)$ to the point $(1, 1)$
\begin{quote}
\begin{verbatim}
from pyx import *
p = path.path(path.moveto(0, 0), path.lineto(1, 1))
\end{verbatim}
\end{quote}
Equivalently, one can also use the predefined \verb|path| subclass
\verb|line| and write
\begin{quote}
\begin{verbatim}
p = path.line(0, 0, 1, 1)
\end{verbatim}
\end{quote}

Usually, one wants to draw the path on a canvas, which can be done
using the \verb|stroke| method of the \verb|canvas| class, i.e.,
\begin{quote}
\begin{verbatim}
c = canvas.canvas()
c.stroke(p)
c.writeEPSfile("line")
\end{verbatim}
\end{quote}
To complete the example, we have added a \verb|writeEPSfile| call,
which writes the contents of the canvas into the given file.

Let us as second example define a path which consists of more than 
one sub path:
\begin{quote}
\begin{verbatim}
cross = path.path(path.moveto(0, 0), path.rlineto(1, 1),
                  path.moveto(1, 0), path.rlineto(-1, 1))
\end{verbatim}
\end{quote}
The first sub path is again a straight line from $(0, 0)$ to $(1, 1)$,
with the only difference that we now have used the \verb|rlineto|
class, whose arguments count relative from the last point in the path.
The second \verb|moveto| instance opens a new sub path starting at the
point $(1, 0)$ and ending at $(0, 1)$. Note that although both lines
intersect at the point $(1/2, 1/2)$, they count as separate sub paths.
The general rule is that each occurence of \verb|moveto| opens a new
sub path. This means that if one wants to draw a rectangle, one should
not use
\begin{quote}
\begin{verbatim}
# wrong: do not use moveto when you want a single sub path
rect1 = path.path(path.moveto(0, 0), path.lineto(0, 1),
                  path.moveto(1, 0), path.lineto(1, 1),
                  path.moveto(1, 1), path.lineto(1, 1),
                  path.moveto(0, 1), path.lineto(0, 0))
\end{verbatim}
\end{quote}
which would construct a rectangle consisting of four disconnected
sub paths. Instead the correct way of defining a rectangle is 
\begin{quote}
\begin{verbatim}
# correct: a rectangle consisting of a single closed sub path
rect2 = path.path(path.moveto(0, 0), path.lineto(0, 1), 
                  path.lineto(1, 1), path.lineto(1, 0),
                  path.closepath())
\end{verbatim}
\end{quote}
%
\begin{figure}
\centerline{\includegraphics{rects}}
\caption{Not closed (left) and closed (midlle) rectangle. Filling a
  path (right) always closes it automatically.}
\label{fig:rects}
\end{figure}
Note that for the last straight line of the rectangle (from $(0, 1)$
back to the origin at $(0, 0)$)) we have used \verb|closepath|.  This
directive adds a straight line from the current point to the first
point of the current sub path and furthermore \textit{closes} the sub
path, i.e., it joins the beginning and the end of the line segment.
The difference can be appreciated in Fig.~\ref{fig:rects}, where
also a filled (and at the same time stroked) rectangle is shown.
The corresponding code looks like
\begin{quote}
\begin{verbatim}
c.stroke(rect1, [deco.filled([color.grey(0.95)])])
\end{verbatim}
\end{quote}
The important point to remember here is that when filling a path, PostScript
automatically closes it.

Of course, rectangles are also predefined in \PyX{}, so above we could
have as well written
\begin{quote}
\begin{verbatim}
rect2 = path.rect(0, 0, 1, 1)
\end{verbatim}
\end{quote}
Here, the first two arguments specify the origin of the rectangle
while the second two arguments define its width and height,
respectively.

XXX arc, bezier example

\section{Path operations}

Often, one not only wants to stroke or fill a path on the canvas
but before do some geometrical operations with it. For instance, one
might want to intersect one path with another one and the split the
paths at the intersection points and then join the segments together
in a new way. \PyX{} supports such tasks by means of a number
of path methods, which we will introduce in the following.

Suppose you want to draw the radii to the intersection points of a
circle with a straight line. This can be done with the following code
which gives the result shown in Fig.~\ref{fig:radii}
\begin{quote}
\begin{verbatim}
from pyx import *

c = canvas.canvas()

circ = path.circle(0, 0, 5)
line = path.line(-7, 2, 7, 5)
c.stroke(circ)
c.stroke(line)

isects = circ.intersect(line)[0]
for isect in isects:
    c.stroke(path.line(0, 0, *circ.at(isect)))

c.writeEPSfile("radii")
\end{verbatim}
\end{quote}
\begin{figure}
\centerline{\includegraphics{radii}}
%\caption{}
\label{fig:rects}
\end{figure}

XXX continue


\section{Class pathel}

The class \verb|pathel| is the superclass of all PostScript path
construction primitives. It is never used directly, but only by
instantiating its subclasses, which correspond one by one to the
PostScript primitives.

\medskip
\begin{tabularx}{\linewidth}{>{\hsize=.7\hsize}X>{\raggedright\arraybackslash\hsize=1.3\hsize}X}
Subclass of \texttt{pathel} & function \\
\hline
\texttt{closepath()} & closes current subpath \\
\texttt{moveto(x, y)} & sets current point to (\texttt{x},
\texttt{y})\\
\texttt{rmoveto(dx, dy)} & moves current point by (\texttt{dx},
\texttt{dy})\\
\texttt{lineto(x, y)} & moves current point to (\texttt{x}, \texttt{y})
while drawing a straight line\\
\texttt{rlineto(dx, dy)} & moves current point by by (\texttt{dx}, \texttt{dy})
while drawing a straight line\\
\texttt{arc(x, y, r, \newline\phantom{arc(}angle1, angle2)} & appends arc segment in
counterclockwise direction with center (\texttt{x}, \texttt{y}) and
radius~\texttt{r} from \texttt{angle1} to \texttt{angle2} (in degrees).\\
\texttt{arcn(x, y, r, \newline\phantom{arcn(}angle1, angle2)} & appends arc segment in
clockwise direction with center (\texttt{x}, \texttt{y}) and
radius~\texttt{r} from \texttt{angle1} to \texttt{angle2} (in degrees). \\
\texttt{arct(x1, y1, x2, y2, r)} & appends arc segment of radius \texttt{r}
connecting between (\texttt{x1}, \texttt{y1}) and (\texttt{x2}, \texttt{y2}).\\
\texttt{rcurveto(dx1, dy1, \newline\phantom{rcurveto(}dx2, dy2,\newline\phantom{rcurveto(}dx3, dy3)} & appends a B\'ezier curve with
the following four control points: current point and the points defined 
relative to the current point by (\texttt{dx1}, \texttt{dy1}), 
(\texttt{dx2}, \texttt{dy2}), and (\texttt{dx3}, \texttt{dy3})
\end{tabularx}
\medskip

Some notes on the above:
\begin{itemize}
\item All coordinates are in \PyX\ lengths
\item If the current point is defined before an \verb|arc| or
  \verb|arcn| command, a straight line from current point to the
  beginning of the arc is prepended.
\item The bounding box (see below) of B\'ezier curves is actually 
  the box enclosing the control points, \textit{i.e.}\ not neccesarily the 
  smallest rectangle enclosing the B\'ezier curve.
\end{itemize}


\section{Class path}

The class path represents PostScript like paths in \PyX. The \verb|path| 
constructor allows the creation of such a path out of a series of 
\verb|pathel|s. The following simple example generates a triangle:
looks like:
\begin{quote}
\begin{verbatim}
from pyx import *
from pyx.path import *

p = path(moveto(0, 0), 
         lineto(0, 1),
         lineto(1, 1),
         closepath())
\end{verbatim}
\end{quote}
In section~\ref{chap:canvas}, we shall see, how it is possible to output such 
a path on a canvas. For the moment, we only want to discuss the methods 
provided by the \verb|path| class. These range from standard operations like 
the determination of the length of a path via \verb|len(p)|, fetching of
items using \verb|p[index]| and the possibility to concatenate two
paths, \verb|p1 + p2|, append further \verb|pathel|s using
\verb|p.append(pathel)| to more advanced methods, which are summarized
in the following table.

XXX terminology: subpath, \dots

\medskip
\begin{tabularx}{\linewidth}{>{\hsize=.7\hsize}X>{\raggedright\arraybackslash\hsize=1.3\hsize}X}
  \texttt{path} method & function \\
  \hline \texttt{\_\_init\_\_(*pathels)} & construct new \texttt{path}
  consisting of \texttt{pathels}\\
  \texttt{append(pathel)} & appends \texttt{pathel} to the end of 
  \texttt{path}\\
  \texttt{arclen(epsilon=1e-5)} & returns the total arc length of
  all \texttt{path} segments in PostScript points with accuracy
  \texttt{epsilon}.$^\dagger$\\
  \texttt{at(t)} & returns the coordinates of the point of
  \texttt{path} corresponding to the parameter value
  \texttt{t}.$^\dagger$\\
  \texttt{arclentoparam(l, \newline\phantom{lentopar(}epsilon=1e-5)} & returns the
  parameter value corresponding to the lengths \texttt{l} (one or a list of
  lengths). This uses arclen-calculations with accuracy
  \texttt{epsilon}.$^\dagger$\\
  \texttt{bbox()} & returns the bounding box of the \texttt{path}\\
  \texttt{begin()} & return first point of first subpath of
  \texttt{path}.$^\dagger$\\
  \texttt{end()} & return last point of last subpath of
  \texttt{path}.$^\dagger$\\
  \texttt{glue(opath)} & returns the \texttt{path} glued together with
  \texttt{opath}, \textit{i.e.}\ the last subpath of \texttt{path}
  and the first one of \texttt{opath} are joined.$^\dagger$\\
  \texttt{intersect(opath, \newline\phantom{intersect(}epsilon=1e-5)}
  & returns tuple consisting of two lists of parameter values
  corresponding to the
  intersection points of \texttt{path} and \texttt{opath}, respectively.$^\dagger$\\
  \texttt{reversed()} & returns the normalized reversed
  \texttt{path}.$^\dagger$\\
  \texttt{split(parameters)} & splits the path at the given list of
  parameters (which have to be sorted in ascending order) and returns
  a corresponding list of 
  \texttt{normpath}s.$^\dagger$\\
    \texttt{tangent(t, length=None)} & return the tuple corresponding to
    the tangent vector to the path at the parameter value \texttt{t}.
    Negative values of \texttt{t} count
    from the end of the path. The absolute value of \texttt{t} must be
    smaller or equal to the number of segments in the normpath,
    otherwise None is returned.  At discontinuities in the path, the
    limit from below is returned. If \texttt{length} is not
  \texttt{None}, the tangent vector will be scaled correspondingly.
  \\
  \texttt{transformed(trafo)} & returns the normalized and accordingly
  to the linear transformation \texttt{trafo} transformed path. Here,
  \texttt{trafo} must be an instance of the \texttt{trafo.trafo}
  class.$^\dagger$
\end{tabularx} 
\medskip

Some notes on the above:
\begin{itemize}
\item The bounding box may be too large, if the path contains any
  \texttt{curveto} elements, since for these the control box,
  \textit{i.e.}, the bounding box enclosing the control points of
  the B\'ezier curve is returned.
\item The $\dagger$ denotes methods which require a prior
  conversion of the path into a \verb|normpath| instance. This is
  done automatically, but if you need to call such methods often,
  it is a good idea to do the conversion once for performance reasons.
\item Instead of using the \verb|glue| method, you can also glue two
paths together with help of the \verb|<<| operator, for instance
\verb|p = p1 << p2|.
\end{itemize}

\section{Class normpath}

The \texttt{normpath} class represents a specialized form of a
\texttt{path} containing only the elements \verb|moveto|,
\verb|lineto|, \verb|curveto| and \verb|closepath|. Such normalized
paths are used during all of the more sophisticated path operations
which are denoted by a $\dagger$ in the above table.


Any path can
easily be converted to its normalized form by passing it as parameter
to the \texttt{normpath} constructor,
\begin{quote}
\begin{verbatim}
np = normpath(p)
\end{verbatim}
\end{quote}
Alternatively, by passing a series of \texttt{pathel}s to the constructor, a
\texttt{normpath} can be constructed like a generic \texttt{path}.
The sum of a \verb|normpath| and a \verb|path| always yields a
\verb|normpath|.

\section{Subclasses of path}

For your convenience, some special PostScript paths are already defined, which
are given in the following table.

\medskip
\begin{tabularx}{\linewidth}{l>{\raggedright\arraybackslash}X}
Subclass of \texttt{path} & function \\
\hline
\texttt{line(x1, y1, x2, y2)} & a line from the point
  (\texttt{x1}, \texttt{y1}) to the point (\texttt{x2}, \texttt{y2})\\
\texttt{curve(x0, y0, x1, y1, x2, y2, x3, y3)} & a B\'ezier curve with 
control points  (\texttt{x0}, \texttt{y0}), $\dots$, (\texttt{x3}, \texttt{y3}).\\
\texttt{rect(x, y, w, h)} &  a rectangle with the
  lower left point (\texttt{x}, \texttt{y}), width~\texttt{w}, and
  height~\texttt{h}. \\
\texttt{circle(x, y, r)} & a circle with 
  center (\texttt{x}, \texttt{y}) and radius~\texttt{r}.
\end{tabularx}
\medskip
Note that besides the \verb|circle| class all classes are actually
subclasses of \verb|normpath|.


% \section{Examples}



%%% Local Variables:
%%% mode: latex
%%% TeX-master: "manual.tex"
%%% End:
