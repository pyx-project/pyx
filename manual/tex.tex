\section{\TeX/\LaTeX{} interface}
\label{tex}
Text in \PyX{} is created by \TeX{} or \LaTeX. From the technical point
of view, the text is inserted as an eps-file. This epsfile is generated
by the module \verb|tex| which runs \TeX{} or \LaTeX{} followed by
\verb|dvips| to create the requested text. \TeX-runs are performed by
instances of the class \verb|tex.tex| while \LaTeX-runs are done by
\verb|tex.latex|. Up to the constructor and the advanced possibilities
for the commands performed in \LaTeX{} both classes \verb|tex.tex| and
\verb|tex.latex| are identical. They provide 5 methods to the user
listed in the following table:

\bigskip
\begin{tabularx}{\linewidth}{ll>{\raggedright\arraybackslash}X}
method&task&attributes\\
\hline
\texttt{text(x, y, cmd, *style)}&perform text output&\texttt{style, fontsize, halign, valign, direction, color, msghandler}\\
\texttt{define(cmd, *style)}&macro definitions&\texttt{msghandler}\\
\texttt{textwd(cmd, *style)}&get text width&\texttt{style, fontsize, msghandler}\\
\texttt{textht(cmd, *style)}&get text height&\texttt{style, fontsize, valign, msghandler}\\
\texttt{textdp(cmd, *style)}&get text depth&\texttt{style, fontsize, valign, msghandler}\\
\end{tabularx}
\bigskip

There are some common rules:
\begin{itemize}
\item \verb|define| can only called before any other of the methods.
\item The extent routines \verb|textwd|, \verb|textht|, and
\verb|textdp| return true \PyX{} length (see section~\ref{length})
and return zero at the first time of requesting the extent. The
evaluation takes place when performing a write. Therefore you have to
run your file twice at first to get the correct number.
\item Commands are done in the order they are called with one exception:
if the same text output is requested more then once (calculating extents
of the command included), all requests are performed at the position of
the first occurrence.
\item All text is inserted into the \verb|canvas| at the position, where
the \verb|tex|- or \verb|latex|-instance itself is inserted into the
\verb|canvas|.
\end{itemize}

The tailing \verb|*style| parameter in all the commands stands for a
list of attribute parameters listed in the last column. Attribute
parameters are instances of classes discussed in detail in the following
section.

\subsection{Attributes}
\begin{description}
\item[\texttt{style:}] \verb|tex.style.text| (default -- does nothing to
the command),\\\verb|tex.style.math| (switches to math mode in
\verb|\displaystyle|)
\item[\texttt{fontsize:}] specifies the \LaTeX-fontsizes by
\verb|tex.fontsize.xxx| where \verb|xxx| is
\verb|tiny|,
\verb|scriptsize|,
\verb|footnotesize|,
\verb|small|,
\verb|normalsize| (default),
\verb|large|,
\verb|Large|,
\verb|LARGE|,
\verb|huge|, or
\verb|Huge|.
\item[\texttt{halign:}] \verb|tex.halign.left| (default),
\verb|tex.halign.center|, \verb|tex.halign.right|
\item[\texttt{valign:}] \verb|tex.valign.top(length)| or
\verb|tex.valign.bottom(length)| --- creates a vertical box with width
\verb|length|. The vertical alignment is the baseline of the first line
for \verb|top| and the last line for \verb|bottom|. The box width is
stored in the \TeX{} dimension \verb|\linewidth|.
\item[\texttt{direction:}] \verb|tex.direction.xxx| where \verb|xxx| is
one of \verb|horizontal| (default), \verb|vertical|, \verb|upsidedown|,
or \verb|rvertical|. Additionally, any angle \verb|angle| (in degree) is
allowed in \verb|tex.direction(angle)|.
\item[\texttt{color:}] stands for any \PyX{} color (see
section~\ref{color}), default is \verb|color.grey.black|
\item[\texttt{msghandler:}] provides a filter for \TeX{} and \LaTeX{}
messages and defines, which messages are hidden. In the following table
the predefined message handlers are described.

\bigskip
\begin{tabularx}{\linewidth}{l>{\raggedright\arraybackslash}X}
msghandler&description\\
\hline
\texttt{msghandler.showall}&shows all messages\\
\texttt{msghandler.hideload}&Hides messages which are written when loading
packages and including other files. They look like \texttt{(file...)}
where \texttt{file} is a readable file and \texttt{...} stands for any
text.\\
\texttt{msghandler.hidegraphicsload}&Hides messages which are written by
\texttt{includegraphics} of the \texttt{graphicx} package. They look like
\texttt{<file>} where \texttt{file} is a readable file.\\
\texttt{msghandler.hidefontwarning}&Hides \LaTeX{} font warnings. They
look like \texttt{LaTeX Font Warning:} and are followed by lines starting
with \texttt{(Font)}.\\
\texttt{msghandler.hidebuterror}&Hides all messages except those
containing a line which starts with ``\texttt{! }''.\\
\texttt{msghandler.hideall}&hides all messages\\
\texttt{msghandler.combine(*list)}&combines msghandlers in
\texttt{*list} by applying them sequentially\\
\end{tabularx}
\end{description}

\subsection{Examples}
\subsubsection{Example 1}
\begin{verbatim}
import pyx
c=canvas.canvas()
t=c.insert(tex.tex())
t.text(0, 0, "Hello, world!")
print "width:", t.textwd("Hello, world!")
print "height:", t.textht("Hello, world!")
print "depth:", t.textdp("Hello, world!")
c.writetofile("tex_ex1")
\end{verbatim}

The output of this program is:
\begin{verbatim}
width: (0.019535 t + 0.000000 u + 0.000000 v + 0.000000 w) m
height: (0.002441 t + 0.000000 u + 0.000000 v + 0.000000 w) m
depth: (0.000683 t + 0.000000 u + 0.000000 v + 0.000000 w) m
\end{verbatim}

The file \verb|tex_ex1.eps| is created and looks like:

\includegraphics{tex_ex1}

\subsubsection{Example 2}
\begin{verbatim}
import pyx
c=canvas.canvas()
t=c.insert(tex.tex())
t.text(0, 0, "Hello, world!")
t.text(0, -0.5, "Hello, world!", tex.fontsize.large)
t.text(0, -1.5,
       r"\sum_{n=1}^{\infty} {1\over{n^2}} = {{\pi^2}\over 6}",
       tex.style.math)

c.draw(path.line(5, -0.5, 9, -0.5))
c.draw(path.line(5, -1, 9, -1))
c.draw(path.line(5, -1.5, 9, -1.5))
c.draw(path.line(7, -1.5, 7, 0))
t.text(7, -0.5, "left aligned") # default is tex.halign.left
t.text(7, -1, "center aligned", tex.halign.center)
t.text(7, -1.5, "right aligned", tex.halign.right)

c.draw(path.line(0, -4, 2, -4))
c.draw(path.line(0, -2.5, 0, -5.5))
c.draw(path.line(2, -2.5, 2, -5.5))
t.text(0, -4,
       "a b c d e f g h i j k l m n o p q r s t u v w x y z",
       tex.valign.top(2))
c.draw(path.line(2.5, -4, 4.5, -4))
c.draw(path.line(2.5, -2.5, 2.5, -5.5))
c.draw(path.line(4.5, -2.5, 4.5, -5.5))
t.text(2.5, -4,
       "a b c d e f g h i j k l m n o p q r s t u v w x y z",
       tex.valign.bottom(2))

c.draw(path.line(5, -4, 9, -4))
c.draw(path.line(7, -5.5, 7, -2.5))
t.text(7, -4, "horizontal")
t.text(7, -4, "vertical", tex.direction.vertical)
t.text(7, -4, "rvertical", tex.direction.rvertical)
t.text(7, -4, "upsidedown", tex.direction.upsidedown)
t.text(7.5, -3.5, "45", tex.direction(45))
t.text(6.5, -3.5, "135", tex.direction(135))
t.text(6.5, -4.5, "225", tex.direction(225))
t.text(7.5, -4.5, "315", tex.direction(315))

t.text(0, -6, "red", color.rgb.red)
t.text(3, -6, "green", color.rgb.green)
t.text(6, -6, "blue", color.rgb.blue)
c.writetofile("tex_ex2")
\end{verbatim}

The file \verb|tex_ex2.eps| is created and looks like:

\includegraphics{tex_ex2}

\subsection{Implementation details}
to be written

\subsection{Known bugs}
\begin{itemize}
\item The end of the last paragraph in a vertical box
(\verb|tex.valign.top| and \verb|tex.valign.bottom|) must be explictly
written (by the command \verb|\par| or an empty line) when a paragraph
formating parameter is changed locally (like \verb|\baselineskip| or a
font size change).  Otherwise, the information is thrown away due to
closing of the box before the paragraph formatting is performed.
\item Due to \verb|dvips| the bounding box is wrong for rotated text.
The rotation is just ignored in the bounding box calculation.
\end{itemize}
