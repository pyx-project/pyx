\chapter{Axes\label{axis}}

Axes are a fundamental component of graphs although there might be use
cases outside of the graph system. Internally axes are constructed out
of components, which handle different tasks and axis need to fullfill:

\begin{definitions}
\term{axis}
  Basically a container for axis data and the components. It
  implements the conversion of a data value to a graph coordinate of
  range [0:1]. It does also handle the proper usage of the components
  in complicated tasks (\emph{i.e.} combine the partitioner, texter,
  painter and rater to find the best partitioning).
\term{tick}
  Ticks are plotted along the axis. They might be labeled with text as
  well.
\term{partitioner, in the code the short form ``parter'' is used}
  Creates one or several choises of tick lists suitable to a certain
  axis range.
\term{texter}
  Creates labels for ticks when they are not set manually.
\term{painter}
  Responsible to paint the axis.
\term{rater}
  Calculate ratings, which can be used to select the best suitable
  partitioning.
\end{definitions}

The names above map directly to modules, which are provided in the
directory \file{graph/axis}. Sometimes it might be conventient to
import the axis directory directly rather access them through the
graph. This would look like:
\begin{verbatim}
  from pyx import *
  graph.axis.painter() # and the like

  from pyx.graph import axis
  axis.painter() # this is shorter ...
\end{verbatim}

In most cases different implementations are available through
different classes, which can be combined in various ways. There are
various axis examples distributed with \PyX{}, where you can see some
of the features of the axis with a few lines of code each. Hence we
can here directly step on to the reference of the available
components.

\section{Axes}

\declaremodule{}{graph.axis.axis}
\modulesynopsis{Axes}

The following classes are part of the module \module{graph.axis.axis}.
However, there is a shortcut to access those classes via
\code{graph.axis} directly. Instances of the classes can be passed to
the \var{**axes} keyword arguments of a graph. Those instances should
be used once only.

\begin{classdesc}{linear}{min=None, max=None, reverse=0, divisor=None, title=None,
                          parter=parter.autolinear(), manualticks=[],
                          density=1, maxworse=2, rater=rater.linear(),
                          texter=texter.default(), painter=painter.plain()}
  This class provides a linear axis. \var{min} and \var{max} are the
  axis range. When not set, they are adjusted automatically by the
  data to be plotted in the graph. Note, that some data might want to
  access the range of an axis (\emph{e.g.} the \class{function} class
  when no range was provided there) or you need to specify a range
  when using the axis without plugging it into a graph (\emph{e.g.}
  when drawing a axis along a path).

  \var{reverse} can be set to indicate a reversed axis starting with
  bigger values first. Alternatively you can fix the axis range by
  \var{min} and \var{max} accordingly. When divisor is set, it is
  taken to divide all data range and position informations while
  creating ticks. You can create ticks not taking into account a
  factor by that. \var{title} is the title of the axis.

  \var{parter} is a partitioner instance, which creates suitable ticks
  for the axis range. Those ticks are merged with manual given ticks
  by \var{manualticks} before proceeding with rating, painting
  \emph{etc.} Manually placed ticks win against those created by the
  partitioner. For automatic partitioners, which are able to calculate
  several possible tick lists for a given axis range, the
  \var{density} is a (linear) factor to favour more or less ticks. It
  should not be stressed to much (its likely, that the result would be
  unappropriate or not at all valid in terms of rating label
  distances). But within a range of say 0.5 to 2 (even bigger for
  large graphs) it can help to get less or more ticks than the default
  would lead to. \var{maxworse} is a the number of trials with more
  and less ticks when a better rating was already found. \var{rater}
  is a rater instance, which rates the ticks and the label distances
  for being best suitable. It also takes into account \var{density}.
  The rater is only needed, when the partitioner creates several tick
  lists.

  \var{texter} is a texter instance. It creates labels for those
  ticks, which claim to have a label, but do not have a label string
  set already. Ticks created by partitioners typically receive their
  label strings by texters. The \var{painter} is finally used to
  construct the output. Note, that usually several output
  constructions are needed, since the rater is also used to rate the
  distances between the label for an optimum.
\end{classdesc}

\begin{classdesc}{lin}{...}
  This class is an abbreviation of \class{linear} described above.
\end{classdesc}

\begin{classdesc}{logarithmic}{min=None, max=None, reverse=0, divisor=None, title=None,
                               parter=parter.autologarithmic(), manualticks=[],
                               density=1, maxworse=2, rater=rater.logarithmic(),
                               texter=texter.default(), painter=painter.plain()}
  This class provides a logarithmic axis. All parameters work like
  \class{linear}. Only two parameters have a different default:
  \var{parter} and \var{rater}. Furthermore and most importantly, the
  mapping between data and graph coordinates is logarithmic.
\end{classdesc}

\begin{classdesc}{log}{...}
This class is an abbreviation of \class{logarithmic} described above.
\end{classdesc}

\begin{classdesc}{linked}{linkedaxis, painter=painter.linked()}
  This class provides an axis, which is linked to another axis
  instance. This means, it shares all its properties with the axis it
  is linked too except for the painter. Thus a linked axis is painted
  differently.

  A standard use case are the \code{x2} and \code{y2} axes in an
  x-y-graph. Linked axes to the \code{x} and \code{y} axes are created
  automatically when not disabled by setting those axes to
  \code{None}. By that, ticks are stroked at both sides of an
  x-y-graph. However, linked axes can be used for in other cases as
  well. You can link axes within a graph or between different graphs
  as long as the orgininal axis is finished first (it must fix its
  layout first).
\end{classdesc}

\begin{classdesc}{split}{subaxes, splitlist=[0.5],
                         splitdist=0.1, relsizesplitdist=1,
                         title=None, painter=painter.split()}
  This class provides an axis, splitting the input values to its
  subaxes depeding on the range of the subaxes. Thus the subaxes
  need to have fixed range, up to the minimum of the first axis and
  the maximum of the last axis. \var{subaxes} actually takes the list
  of subaxes. \var{splitlist} defines the positions of the spliting
  in graph coordinates. Thus the length of \var{subaxes} must be the
  length of \var{splitlist} plus one. If an entry in \var{splitlist}
  is \code{None}, the axes aside define the split position taking into
  account the ratio of the axes ranges (meassured by an internal
  \code{relsize} attribute of each axis).

  \var{splitdist} is the space reserved for a splitting in graph
  coordinates, when the corresponding entry in \var{splitlist} is not
  \code{None}. \var{relsizesplitdist} is the space reserved for the
  splitting in terms, when the corresponding entry in \var{splitlist}
  is \code{None} compared to the \code{relsize} of the axes aside.

  \var{title} is the title of the split axes and \var{painter} is a
  specialized painter, which takes care of marking the axes breaks,
  while the painting of the subaxes are performed by their painters
  themself.
\end{classdesc}

\begin{classdesc}{linkedsplit}{linkedaxis,
                               painter=painter.linkedsplit(),
                               subaxispainter=omitsubaxispainter}
  This class provides an axis, which is linked to an instance of
  \class{split}. The purpose of a linked axis is described in class
  \class{linked} above. \var{painter} replaces the painter from the
  \var{linkedaxis} instance.

  While this class creates linked axes for the subaxes of
  \var{linkedsplit} as well, the question arises what painters to use
  there. When \var{subaxispainter} is not set, no painter is given
  explicitly leaving this decision to the subaxes themself. This will
  lead to omitting all labels and the title. However, you can use a
  changeable attribute of painters in \var{subaxispainter} to replace
  the default.
\end{classdesc}

\begin{classdesc}{bar}{subaxis=None, multisubaxis=None,
                       dist=0.5, firstdist=None, lastdist=None,
                       title=None, painter=painter.bar()}
  This class provides an axis suitable for a bar style. It handles a
  discrete set of values and maps them to distinct ranges in graph
  coordinates. For that, the axis gets a list as data values. The
  first entry is taken to be one of the discrete values valid on this
  axis. All other parameters, lets call them others, are passed to a
  subaxis. When others has only one entry, it is passed as a value,
  otherwise as a list. The result of the conversion done by the
  subaxis is mapped into the graph coordinate range for this discrete
  value. When neigher \var{subaxis} nor \var{multisubaxis} is set,
  others must be a single value in range [0:1]. This value is used for
  the position at the subaxis without converion.

  When \var{subaxis} is set, it is used for the conversion of others.
  When \var{multisubaxis} is set, it must be an instance of \var{bar}
  as well. It is than dublicated for each of the discrete values
  allowed for the axis. By that, you can create nested bar axes with
  a different discrete values for each discrete value of the axis. It
  is not allowed to set both, \var{subaxis} and \var{multisubaxis}.

  \var{dist} is used as the spacing between the ranges for each
  distinct value. It is measured in the same units as the subaxis
  results, thus the default value of \code{0.5} means halve the width
  between the distinct values as the width for each distinct value.
  \var{firstdist} and \var{lastdist} are used before the first and
  after the last value. When set to \code{None}, halve of \var{dist}
  is used.

  \var{title} is the title of the split axes and \var{painter} is a
  specialized painter for an bar axis. When \var{multisubaxis} is
  used, their painters are called as well, otherwise they are not
  taken into account.
\end{classdesc}

\section{Ticks}

\declaremodule{}{graph.axis.tick}
\modulesynopsis{Axes ticks}

We describe only one class of the module \module{graph.axis.tick}
here. Instances of this class are usefill in the \code{manualticks}
parameter of a regular axis.

\begin{classdesc}{tick}{pos, ticklevel=0, labellevel=0, label=None,
                        labelattrs=[], power=1, floatprecision=10}
  \var{pos} is the position of the tick. Since ticks use rational
  number arithmetics, a convertion to a rational number needs to be
  performed. \var{pos} can hold different types for that:
  \begin{itemize}
  \item A float. It is converted to a rational with finite precision
    determined by \var{floatprecision}.
  \item A string, which is parsed to a rational number with full
    precision. It is also allowed to provide a fraction like
    \samp{1/3}.
  \item A tuple of two integers. Those integers are taken as
    enumerator and denominator of the rational.
  \end{itemize}

  A tick has a tick level (\emph{i.e.} markers at the axis path) and a
  label lavel (\emph{e.i.} place text at the axis path),
  \var{ticklevel} and \var{labellevel}. These are non-negative
  integers or \var{None}. A value of \code{0} means a regular tick or
  label, \code{1} stands for a subtick or sublabel, \code{2} for
  subsubtick or subsublabel and so on. \code{None} means omitting the
  tick or label. \var{label} is the text of the label. When not set,
  it can be created automatically by a texter. \var{labelattrs} are
  the attributes for the labels.

  \var{power} is an integer to calculate \samp{pos**code}. This is
  usefull at certain places in partitioners.
\end{classdesc}

\section{Partitioners}

\declaremodule{}{graph.axis.parter}
\modulesynopsis{Axes partitioners}

The following classes are part of the module \module{graph.axis.parter}.
Instances of the classes can be passed to the parter keyword argument
of regular axes.

\begin{classdesc}{linear}{tickdist=None, labeldist=None,
                          extendtick=0, extendlabel=None,
                          epsilon=1e-10}
  TODO: to be continued ...
\end{classdesc}

\begin{classdesc}{lin}{...}
This class is an abbreviation of \class{linear} described above.
\end{classdesc}

\begin{classdesc}{autolinear}{variants=defaultvariants,
                              extendtick=0,
                              epsilon=1e-10}
  TODO: to be continued ...
\end{classdesc}

\begin{classdesc}{autolin}{...}
This class is an abbreviation of \class{autolinear} described above.
\end{classdesc}

\begin{classdesc}{logarithmic}{tickpos=None, labelpos=None,
                               extendtick=0, extendlabel=None,
                               epsilon=1e-10}
  TODO: to be continued ...
\end{classdesc}

\begin{classdesc}{log}{...}
This class is an abbreviation of \class{logarithmic} described above.
\end{classdesc}

\begin{classdesc}{autologarithmic}{variants=defaultvariants,
                                   extendtick=0, extendlabel=None,
                                   epsilon=1e-10}
  TODO: to be continued ...
\end{classdesc}

\begin{classdesc}{autolog}{...}
This class is an abbreviation of \class{autologarithmic} described above.
\end{classdesc}

