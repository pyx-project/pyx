\documentclass{manual}
\usepackage{pyx}
\usepackage{graphicx}
\usepackage[T1]{fontenc}

\title{\PyX{} Reference Manual}
\author{J\"org Lehmann, Andr\'e Wobst}
\authoraddress{http://pyx.sourceforge.net/}
\date{DUMMY}
\release{DUMMY}

\makeindex

\begin{document}

\maketitle
\cleardoublepage

\begin{abstract}
\noindent
\PyX{} is a Python package for the creation of PostScript and PDF files. It
combines an abstraction of the PostScript drawing model with a TeX/LaTeX
interface. Complex tasks like 2d and 3d plots in publication-ready quality are
built out of these primitives.
\end{abstract}

\tableofcontents

% \chapter{Introduction}
\label{intro}

\PyX{} is a Python package for the creation of vector graphics. As
such it readily allows one to generate encapsulated PostScript files
by providing an abstraction of the PostScript graphics model.  Based
on this layer and in combination with the full power of the Python
language itself, the user can just code any complexity of the figure
wanted. \PyX{} distinguishes itself from other similar solutions by
its \TeX{}/\LaTeX{} interface that enables one to make direct use of
the famous high quality typesetting of these programs.

A major part of \PyX{} on top of the already described basis is the
provision of high level functionality for complex tasks like 2d plots
in publication-ready quality.

\section{Organisation of the \PyX{} package}

The \PyX{} package is split in several modules, which can be
categorised in the following groups

\begin{tableii}{l|l}{textrm}{Functionality}{Modules}
  basic graphics functionality &   \module{canvas}, \module{path}, \module{deco}, \module{style}, \module{color},
  and \module{connector}
  \\
  text output via \TeX{}/\LaTeX{} &   \module{text} and \module{box}
  \\
  linear transformations and units &   \module{trafo} and \module{unit}
  \\
  graph plotting functionality &  \module{graph} (including submodules)
  and \module{graph.axis} (including submodules)
  \\
  EPS file inclusion & \module{epsfile}
\end{tableii}

These modules (and some other less import ones) are imported into the
module namespace by using 
\begin{verbatim}
from pyx import *
\end{verbatim}
at the beginning of the Python program.  However, in order to prevent
namespace pollution, you may also simply use \samp{import pyx}.
Throughout this manual, we shall always assume the presence of the
above given import line.a



%%% Local Variables:
%%% mode: latex
%%% TeX-master: "manual.tex"
%%% ispell-dictionary: "british"
%%% End:

% \chapter{Basic graphics}

\sectionauthor{J\"org Lehmann}{joergl@users.sourceforge.net} 

\label{graphics}

\section{Introduction}

The path module allows one to construct PostScript-like
\textit{paths}, which are one of the main building blocks for the
generation of drawings. A PostScript path is an arbitrary shape
consisting of straight lines, arc segments and cubic B\'ezier curves.
Such a path does not have to be connected but may also comprise
several disconnected segments, which will be called \textit{subpaths}
in the following.

XXX example for paths and subpaths

Usually, a path is constructed by passing a list of the path
primitives \class{moveto}, \class{lineto}, \class{curveto}, etc., to the
constructor of the \class{path} class. The following code snippet, for
instance, defines a path \var{p} that consists of a straight line
from the point $(0, 0)$ to the point $(1, 1)$
\begin{verbatim}
from pyx import *
p = path.path(path.moveto(0, 0), path.lineto(1, 1))
\end{verbatim}
Equivalently, one can also use the predefined \class{path} subclass
\class{line} and write
\begin{verbatim}
p = path.line(0, 0, 1, 1)
\end{verbatim}

While already some geometrical operations can be performed with this
path (see next section), another \PyX{} object is needed in order to
actually being able to draw the path, namely an instance of the
\class{canvas} class. By convention, we use the name \var{c} for this
instance:
\begin{verbatim}
c = canvas.canvas()
\end{verbatim}
In order to draw the path on the canvas, we use the \method{stroke()} method
of the \class{canvas} class, i.e.,
\begin{verbatim}
c.stroke(p)
c.writeEPSfile("line")
\end{verbatim}
To complete the example, we have added a \method{writeEPSfile()} call,
which writes the contents of the canvas to the file \file{line.eps}.
Note that an extension \file{.eps} is added automatically, if not
already present in the given filename.

As a second example, let us define a path which consists of more than 
one subpath:
\begin{verbatim}
cross = path.path(path.moveto(0, 0), path.rlineto(1, 1),
                  path.moveto(1, 0), path.rlineto(-1, 1))
\end{verbatim}
The first subpath is again a straight line from $(0, 0)$ to $(1, 1)$,
with the only difference that we now have used the \class{rlineto}
class, whose arguments count relative from the last point in the path.
The second \class{moveto} instance opens a new subpath starting at the
point $(1, 0)$ and ending at $(0, 1)$. Note that although both lines
intersect at the point $(1/2, 1/2)$, they count as disconnected
subpaths.  The general rule is that each occurence of a \class{moveto}
instance opens a new subpath. This means that if one wants to draw a
rectangle, one should not use
\begin{verbatim}
rect1 = path.path(path.moveto(0, 0), path.lineto(0, 1),
                  path.moveto(0, 1), path.lineto(1, 1),
                  path.moveto(1, 1), path.lineto(1, 0),
                  path.moveto(1, 0), path.lineto(0, 0))
\end{verbatim}
which would construct a rectangle out of four disconnected
subpaths (see Fig.~\ref{fig:rects}a). In a better solution (see
Fig.~\ref{fig:rects}b), the pen is not lifted between the first and
the last point:
%
\begin{figure}
\centerline{\includegraphics{rects}}
\caption{Rectangle consisting of (a) four separate lines, (b) one open
  path, and (c) one closed path. (d) Filling a
  path always closes it automatically.}
\label{fig:rects}
\end{figure}
%
\begin{verbatim}
rect2 = path.path(path.moveto(0, 0), path.lineto(0, 1), 
                  path.lineto(1, 1), path.lineto(1, 0))
\end{verbatim}
However, as one can see in the lower left corner of
Fig.~\ref{fig:rects}b, the rectangle is still incomplete.  It needs to
be closed, which can  be done explicitly by using for the last straight
line of the rectangle (from the point $(0, 1)$ back to the origin at $(0, 0)$)
the \class{closepath} directive:
\begin{verbatim}
rect3 = path.path(path.moveto(0, 0), path.lineto(0, 1), 
                  path.lineto(1, 1), path.lineto(1, 0),
                  path.closepath())
\end{verbatim}
The \class{closepath} directive adds a straight line from the current
point to the first point of the current subpath and furthermore
\textit{closes} the sub path, i.e., it joins the beginning and the end
of the line segment. This results in the intended rectangle shown in
Fig.~\ref{fig:rects}c. Note that filling the path implicitly closes
every open subpath, as is shown for a single subpath in
Fig.~\ref{fig:rects}d), which results from
\begin{verbatim}
c.stroke(rect2, [deco.filled([color.grey(0.95)])])
\end{verbatim}
More details on the available path elements can be found in
Sect.~\ref{path:pathitem}.

XXX more on styles and attributes and reference to corresponding section

Of course, rectangles are also predefined in \PyX{}, so above we could
have as well written
\begin{verbatim}
rect2 = path.rect(0, 0, 1, 1)
\end{verbatim}
Here, the first two arguments specify the origin of the rectangle
while the second two arguments define its width and height,
respectively. For more details on the predefined paths, we
refer the reader to Sect.~\ref{path:predefined}.

\section{Path operations}

Often, one wants to perform geometrical operations with a path before
placing it on a canvas by stroking or filling it.  For instance, one
might want to intersect one path with another one, split the paths at
the intersection points, and then join the segments together in a new
way. \PyX{} supports such tasks by means of a number of path methods,
which we will introduce in the following.

Suppose you want to draw the radii to the intersection points of a
circle with a straight line. This task can be done using the following
code which results in Fig.~\ref{fig:radii}
\verbatiminput{radii.py}
\begin{figure}
\centerline{\includegraphics{radii}}
\caption{Example: Intersection of circle with line yielding two radii.}
\label{fig:radii}
\end{figure}
Here, the basic elements, a circle around the point $(0, 0)$ with
radius $2$ and a straight line, are defined. Then, passing the \var{line}, to
the \method{intersect()} method of \var{circle}, we obtain a tuple of
parameter values of the intersection points. The first element of the
tuple is a list of parameter values for the path whose
\method{intersect()} method has been called, the second element is the
corresponding list for the path passed as argument to this method. In
the present example, we only need one list of parameter values, namely
\var{isects_circle}.  Using the \method{at()} path method to obtain
the point corresponding to the parameter value, we draw the radii for
the different intersection points. 

Another powerful feature of \PyX{} is its ability to split paths at a
given set of parameters. For instance, in order to fill in the
previous example the segment of the circle delimited by the straight
line (cf.\ Fig.~\ref{fig:radii2}), one first has to construct a path
corresponding to the outline of this segment. The following code
snippet yields this \var{segment}
\begin{verbatim}
arc1, arc2 = circle.split(isects_circle)
if arc1.arclen() < arc2.arclen():
    arc = arc1
else:
    arc = arc2

isects_line.sort()
line1, line2, line3 = line.split(isects_line)

segment = line2 << arc
\end{verbatim}
\begin{figure}
\centerline{\includegraphics{radii2}}
\caption{Example: Intersection of circle with line yielding radii and
  circle segment.}
\label{fig:radii2}
\end{figure}
Here, we first split the circle using the \method{split()} method passing
the list of parameters obtained above. Since the circle is closed,
this yields two arc segments. We then use the \method{arclen()}, which
returns the arc length of the path, to find the shorter of the two
arcs. Before splitting the line, we have to take into account that
the \method{split()} method only accepts a sorted list of parameters.
Finally, we join the straight line and the arc segment. For
this, we make use of the \verb|<<| operator, which not only adds
the paths (which could be done using \samp{line2 + arc}), but also
joins the last subpath of \var{line2} and the first one of
\var{arc}. Thus, \var{segment} consists of only a single subpath
and filling works as expected.

An important issue when operating on paths is the parametrisation
used. Internally, \PyX{} uses a parametrisation which uses an interval
of length $1$ for each path element of a path. For instance, for a
simple straight line, the possible parameter values range from $0$ to
$1$, corresponding to the first and last point, respectively, of the
line. Appending another straight line, would extend this range to a
maximal value of $2$. You can always query this maximal value using
the \method{range()} method of the \class{path} class.  

However, the situation becomes more complicated if more complex
objects like a circle are involved. Then, one could be tempted to
assume that again the parameter value range from $0$ to $1$, because
the predefined circle consists just of one \class{arc} together with a
\class{closepath} element. However, as a simple \samp{path.circle(0,
  0, 1).range()} will tell, this is not the case: the actual range is
much larger. The reason for this behaviour lies in the internal path
handling of \PyX: Before performing any non-trivial geometrical
operation with a path, it will automatically be converted into an
instance of the \class{normpath} class (see also
Sect.~\ref{path:normpath}). These so generated paths are already
separated in their subpaths and only contain straight lines and
B\'ezier curve segments. Thus, as is easily imaginable, they are much
simpler to deal with.

A unique way of accessing a point on the path is to use the arc length
of the path segment from the first point of the path to the given
point. Thus, all \PyX{} path methods that accept a parameter value
also allow the user to pass an arc length. For instance, 
\begin{verbatim}
from math import pi

pt1 = path.circle(0, 0, 1).at(arclen=pi)
pt2 = path.circle(0, 0, 1).at(arclen=3*pi/2)

c.stroke(path.path(path.moveto(*pt1), path.lineto(*pt2)))
\end{verbatim}
will draw a straight line from a point at angle $180$ degrees (in
radians $\pi$) to another point at angle $270$ degrees (in radians
$3\pi/2$) on the unit circle.

More information on the available path methods can be found 
in Sect.~\ref{path:path}.

\section{Attributes: Styles and Decorations}

XXX to be done
% \chapter{Basic graphics}

\label{path}

\section{Introduction}

The path module allows one to construct PostScript-like
\textit{paths}, which are one of the main building blocks for the
generation of drawings. A PostScript path is an arbitrary shape built
up of straight lines, arc segments and cubic Bezier curves. Such a
path does not have to be connected but may also consist of multiple
connected segments, which will be called \textit{sub paths} in the
following.

Usually, a path is constructed by passing a list of the path
primitives \verb|moveto|, \verb|lineto|, \verb|curveto|, etc., to the
constructor of the \verb|path| class. The following code snippet, for
instance, defines a path \verb|p| that consists of a straight line
from the point $(0, 0)$ to the point $(1, 1)$
\begin{verbatim}
from pyx import *
p = path.path(path.moveto(0, 0), path.lineto(1, 1))
\end{verbatim}
Equivalently, one can also use the predefined \verb|path| subclass
\verb|line| and write
\begin{verbatim}
p = path.line(0, 0, 1, 1)
\end{verbatim}

While you can already do some geometrical operations with the
just create path (see next section), we need another \PyX{} object
in order to produce the output corresponding to the path, namely
an instance of the \verb|canvas| class. By convetion, we use
the name \verb|c| for this instance:
\begin{verbatim}
c = canvas.canvas()
\end{verbatim}
In order to draw the path on the canvas, we use the \verb|stroke| method
of the \verb|canvas| class, i.e.,
\begin{verbatim}
c.stroke(p)
c.writeEPSfile("line")
\end{verbatim}
To complete the example, we have added a \verb|writeEPSfile| call,
which writes the contents of the canvas into the given file.

Let us as second example define a path which consists of more than 
one sub path:
\begin{verbatim}
cross = path.path(path.moveto(0, 0), path.rlineto(1, 1),
                  path.moveto(1, 0), path.rlineto(-1, 1))
\end{verbatim}
The first sub path is again a straight line from $(0, 0)$ to $(1, 1)$,
with the only difference that we now have used the \verb|rlineto|
class, whose arguments count relative from the last point in the path.
The second \verb|moveto| instance opens a new sub path starting at the
point $(1, 0)$ and ending at $(0, 1)$. Note that although both lines
intersect at the point $(1/2, 1/2)$, they count as separate sub paths.
The general rule is that each occurence of \verb|moveto| opens a new
sub path. This means that if one wants to draw a rectangle, one should
not use
\begin{verbatim}
# wrong: do not use moveto when you want a single sub path
rect1 = path.path(path.moveto(0, 0), path.lineto(0, 1),
                  path.moveto(1, 0), path.lineto(1, 1),
                  path.moveto(1, 1), path.lineto(1, 1),
                  path.moveto(0, 1), path.lineto(0, 0))
\end{verbatim}
which would construct a rectangle consisting of four disconnected
sub paths. Instead the correct way of defining a rectangle is 
\begin{verbatim}
# correct: a rectangle consisting of a single closed sub path
rect2 = path.path(path.moveto(0, 0), path.lineto(0, 1), 
                  path.lineto(1, 1), path.lineto(1, 0),
                  path.closepath())
\end{verbatim}
%
\begin{figure}
\centerline{\includegraphics{rects}}
\caption{Not closed (left) and closed (midlle) rectangle. Filling a
  path (right) always closes it automatically.}
\label{fig:rects}
\end{figure}
Note that for the last straight line of the rectangle (from $(0, 1)$
back to the origin at $(0, 0)$)) we have used \verb|closepath|.  This
directive adds a straight line from the current point to the first
point of the current sub path and furthermore \textit{closes} the sub
path, i.e., it joins the beginning and the end of the line segment.
The difference can be appreciated in Fig.~\ref{fig:rects}, where
also a filled (and at the same time stroked) rectangle is shown.
The corresponding code looks like
\begin{verbatim}
c.stroke(rect1, [deco.filled([color.grey(0.95)])])
\end{verbatim}
The important point to remember here is that when filling a path, PostScript
automatically closes it.

Of course, rectangles are also predefined in \PyX{}, so above we could
have as well written
\begin{verbatim}
rect2 = path.rect(0, 0, 1, 1)
\end{verbatim}
Here, the first two arguments specify the origin of the rectangle
while the second two arguments define its width and height,
respectively.

XXX arc, bezier example

\section{Path operations}

Often, one not only wants to stroke or fill a path on the canvas
but before do some geometrical operations with it. For instance, one
might want to intersect one path with another one and the split the
paths at the intersection points and then join the segments together
in a new way. \PyX{} supports such tasks by means of a number
of path methods, which we will introduce in the following.

Suppose you want to draw the radii to the intersection points of a
circle with a straight line. This task can be done using the following
code which gives the result shown in Fig.~\ref{fig:radii}
\verbatiminput{radii.py}
\begin{figure}
\centerline{\includegraphics{radii}}
%\caption{}
\label{fig:radii}
\end{figure}
Passing another path, here \verb|line|, to the \verb|intersect| method
of \verb|circle|, we obtain a tuple of parameter values of the
intersection points. The first element of the tuple is a list of
parameter values for the path whose \verb|intersect| method we have
called, the second element is the corresponding list for the path
passed as argument to this method. In the present example, we only
need one list of parameter values, namely \verb|isects_circle|.
Iterating over the elements of this list, we draw the radii, using the
\verb|at| path method to obtain the point corresponding to the
parameter value.

Another powerful feature of \PyX{} is its ability to split paths at a
given set of parameters. For instance, in order to fill in the
previous example the segment of the circle delimited by the straight
line (cf.\ Fig.~\ref{fig:radii2}), you first have to construct a path
corresponding to the outline of this segment. The following code
snippet does yield this \verb|segment|
\begin{verbatim}
arc1, arc2 = circle.split(isects_circle)
arc = arc1.arclen()<arc2.arclen() and arc1 or arc2

isects_line.sort()
line1, line2, line3 = line.split(isects_line)

segment = line2 << arc
\end{verbatim}
\begin{figure}
\centerline{\includegraphics{radii2}}
%\caption{}
\label{fig:radii2}
\end{figure}
Here, we first split the circle using the \verb|split| method passing
the list of parameters obtained above. Since the circle is closed,
this yields two arc segments. We then use the \verb|arclen|, which
returns the arc length of the path, to find the shorter of the two
arcs. Before splitting the line, we have to take into account that
the \verb|split| method only accepts a sorted list of parameters.
Finally, we join the straight line and the arc segment. For
this, we make use of the \verb|<<| operator, which not only adds
the paths (which could be done using \verb|line2 + arc|), but also
joins the last sub path of \verb|line2| and the first one of
\verb|arc|. Thus, \verb|segment| consists of only a single sub path
and filling works as expected.

XXX reverse, reversed, parametrisation, arclen parameters




\section{Class pathel}

The class \verb|pathel| is the superclass of all PostScript path
construction primitives. It is never used directly, but only by
instantiating its subclasses, which correspond one by one to the
PostScript primitives.

\medskip
\begin{tabularx}{\linewidth}{>{\hsize=.7\hsize}X>{\raggedright\arraybackslash\hsize=1.3\hsize}X}
Subclass of \texttt{pathel} & function \\
\hline
\texttt{closepath()} & closes current subpath \\
\texttt{moveto(x, y)} & sets current point to (\texttt{x},
\texttt{y})\\
\texttt{rmoveto(dx, dy)} & moves current point by (\texttt{dx},
\texttt{dy})\\
\texttt{lineto(x, y)} & moves current point to (\texttt{x}, \texttt{y})
while drawing a straight line\\
\texttt{rlineto(dx, dy)} & moves current point by by (\texttt{dx}, \texttt{dy})
while drawing a straight line\\
\texttt{arc(x, y, r, \newline\phantom{arc(}angle1, angle2)} & appends arc segment in
counterclockwise direction with center (\texttt{x}, \texttt{y}) and
radius~\texttt{r} from \texttt{angle1} to \texttt{angle2} (in degrees).\\
\texttt{arcn(x, y, r, \newline\phantom{arcn(}angle1, angle2)} & appends arc segment in
clockwise direction with center (\texttt{x}, \texttt{y}) and
radius~\texttt{r} from \texttt{angle1} to \texttt{angle2} (in degrees). \\
\texttt{arct(x1, y1, x2, y2, r)} & appends arc segment of radius \texttt{r}
connecting between (\texttt{x1}, \texttt{y1}) and (\texttt{x2}, \texttt{y2}).\\
\texttt{rcurveto(dx1, dy1, \newline\phantom{rcurveto(}dx2, dy2,\newline\phantom{rcurveto(}dx3, dy3)} & appends a B\'ezier curve with
the following four control points: current point and the points defined 
relative to the current point by (\texttt{dx1}, \texttt{dy1}), 
(\texttt{dx2}, \texttt{dy2}), and (\texttt{dx3}, \texttt{dy3})
\end{tabularx}
\medskip

Some notes on the above:
\begin{itemize}
\item All coordinates are in \PyX\ lengths
\item If the current point is defined before an \verb|arc| or
  \verb|arcn| command, a straight line from current point to the
  beginning of the arc is prepended.
\item The bounding box (see below) of B\'ezier curves is actually 
  the box enclosing the control points, \textit{i.e.}\ not neccesarily the 
  smallest rectangle enclosing the B\'ezier curve.
\end{itemize}


\section{Class path}

The methods provided by instance of the class \verb|path| are
summarized in the following table:

\medskip
\begin{tabularx}{1.04\linewidth}{>{\hsize=.7\hsize}X>{\raggedright\arraybackslash\hsize=1.3\hsize}X}
  \texttt{path} method & function \\
  \hline \texttt{\_\_init\_\_(*pathels)} & construct new \texttt{path}
  consisting of \texttt{pathels}\\
  \texttt{append(pathel)} & appends \texttt{pathel} to the end of 
  \texttt{path}\\
  \texttt{arclen(epsilon=1e-5)} & returns the total arc length of
  all \texttt{path} segments in PostScript points with accuracy
  \texttt{epsilon}.$^\dagger$\\
  \texttt{arclentoparam(lengths, \newline\phantom{arclentoparam(}epsilon=1e-5)} & returns the
  parameter value corresponding to the lengths \texttt{lengths} (one or a list of
  lengths). This uses arclen-calculations with accuracy
  \texttt{epsilon}.$^\dagger$\\
  \texttt{at(param=None,
    \newline\phantom{at(}arclen=None)} & returns the coordinates of the point of
  \texttt{path} corresponding to the parameter value
  \texttt{param} or the arc length \verb|arclen|.$^\dagger$\\
  \texttt{bbox()} & returns the bounding box of the \texttt{path}\\
  \texttt{begin()} & return first point of first subpath of
  \texttt{path}.$^\dagger$\\
   \texttt{curvradius(self, 
    \newline\phantom{curvradius(}param=None,
    \newline\phantom{curvradius(}arclen=None)} &
        Returns the curvature radius (or None if infinite) at parameter param.
        This is the inverse of the curvature at this parameter
        Please note that this radius can be negative or positive,
        depending on the sign of the curvature.
        $^\dagger$
        \\
  \texttt{end()} & return last point of last subpath of
  \texttt{path}.$^\dagger$\\
  \texttt{glue(opath)} & returns the \texttt{path} glued together with
  \texttt{opath}, \textit{i.e.}\ the last subpath of \texttt{path}
  and the first one of \texttt{opath} are joined.$^\dagger$\\
  \texttt{intersect(opath, \newline\phantom{intersect(}epsilon=1e-5)}
  & returns tuple consisting of two lists of parameter values
  corresponding to the
  intersection points of \texttt{path} and \texttt{opath},
  respectively.$^\dagger$\\
  \texttt{range()} & returns the maximal value of the parameter value
  \texttt{param} in the path methods\\
  \texttt{reversed()} & returns the normalized reversed
  \texttt{path}.$^\dagger$\\
  \texttt{split(parameters)} & splits the path at the given list of
  parameters (which have to be sorted in ascending order) and returns
  a corresponding list of 
  \texttt{normpath}s.$^\dagger$\\
  \texttt{tangent(param=None, 
      \newline\phantom{tangent(}arclen=None,
      \newline\phantom{tangent(}length=None)} & return the tuple corresponding to
    the tangent vector to the path at the parameter value
  \texttt{param} or the arc length \texttt{arclen}.
    \texttt{param} (\texttt{arclen}) has to be
    smaller or equal to \texttt{self.range()} (\texttt{self.arclen()}), 
    otherwise an exception is raised.  At discontinuities in the path, the
    limit from below is returned. If \texttt{length} is not
    \texttt{None}, the tangent vector will be scaled correspondingly.
  \\
  \texttt{trafo(param=None, 
      \newline\phantom{trafo(}arclen=None} & return a trafo which maps 
    a point $(0, 1)$ to the tangent vector to the path at the parameter value
    \texttt{param} or the arc length \texttt{arclen}.
    \texttt{param} (\texttt{arclen}) has to be
    smaller or equal to \texttt{self.range()} (\texttt{self.arclen()}), 
    otherwise an exception is raised.  At discontinuities in the path, the
    limit from below is returned.
  \\
  \texttt{transformed(trafo)} & returns the normalized and accordingly
  to the linear transformation \texttt{trafo} transformed path. Here,
  \texttt{trafo} must be an instance of the \texttt{trafo.trafo}
  class.$^\dagger$
\end{tabularx} 
\medskip

Some notes on the above:
\begin{itemize}
\item The bounding box may be too large, if the path contains any
  \texttt{curveto} elements, since for these the control box,
  \textit{i.e.}, the bounding box enclosing the control points of
  the B\'ezier curve is returned.
\item The $\dagger$ denotes methods which require a prior
  conversion of the path into a \verb|normpath| instance. This is
  done automatically, but if you need to call such methods often,
  it is a good idea to do the conversion once for performance reasons.
\item Instead of using the \verb|glue| method, you can also glue two
paths together with help of the \verb|<<| operator, for instance
\verb|p = p1 << p2|.
\item In the methods accepting both a parameter value \verb|param| and
  an arc length \verb|arclen|, exactly one of these arguments has to
  provided.
\end{itemize}

\section{Class normpath}

The \texttt{normpath} class represents a specialized form of a
\texttt{path} containing only the elements \verb|moveto|,
\verb|lineto|, \verb|curveto| and \verb|closepath|. Such normalized
paths are used during all of the more sophisticated path operations
which are denoted by a $\dagger$ in the above table.


Any path can easily be converted to its normalized form by passing it
as parameter to the \texttt{normpath} constructor,
\begin{verbatim}
np = normpath(p)
\end{verbatim}
Additionally, you can specify the accuracy (in points) which is used
in all \verb|normpath| calculations by means of the keyword argument
\verb|epsilon|, which defaults to $10^{-5}$.  Note that the sum of a
\verb|normpath| and a \verb|path| always yields a \verb|normpath|.

\section{Subclasses of path}

For your convenience, some special PostScript paths are already defined, which
are given in the following table.

\medskip
\begin{tabularx}{\linewidth}{l>{\raggedright\arraybackslash}X}
Subclass of \texttt{path} & function \\
\hline
\texttt{line(x1, y1, x2, y2)} & a line from the point
  (\texttt{x1}, \texttt{y1}) to the point (\texttt{x2}, \texttt{y2})\\
\texttt{curve(x0, y0, x1, y1, x2, y2, x3, y3)} & a B\'ezier curve with 
control points  (\texttt{x0}, \texttt{y0}), $\dots$, (\texttt{x3}, \texttt{y3}).\\
\texttt{rect(x, y, w, h)} &  a rectangle with the
  lower left point (\texttt{x}, \texttt{y}), width~\texttt{w}, and
  height~\texttt{h}. \\
\texttt{circle(x, y, r)} & a circle with 
  center (\texttt{x}, \texttt{y}) and radius~\texttt{r}.
\end{tabularx}
\medskip


% \section{Examples}



%%% Local Variables:
%%% mode: latex
%%% TeX-master: "manual.tex"
%%% End:

% \section{Module \module{deformer}}
\label{deformer}

\declaremodule{}{deformer}
\modulesynopsis{Path deformers}

The \module{deformer} module provides techniques to generate modulated paths.
All classes in the \module{deformer} module can be used as attributes when
drawing/stroking paths onto a canvas, but also independently for manipulating
previously created paths. The difference to the classes in the \module{deco}
module is that here, a totally new path is constructed.

All classes of the \module{deformer} module provide the following methods:

\begin{methoddesc}{__call__}{(specific parameters for the class)}
Returns a deformer with modified parameters
\end{methoddesc}

\begin{methoddesc}{deform}{path} Returns the deformed normpath on the basis of
the \var{path}. This method allows using the deformers outside of a
drawing call.
\end{methoddesc}

The deformer classes are the following:

\begin{classdesc}{cycloid}{radius, halfloops=10, skipfirst=1*unit.t_cm,
skiplast=1*unit.t_cm, curvesperhloop=3, sign=1, turnangle=45}
This deformer creates a cycloid around a path. The outcome looks similar to
a 3D spring stretched along the original path.

\var{radius}: the radius of the cycloid (this is the radius of the 3D spring)

\var{halfloops}: the number of half-loops of the cycloid

\var{skipfirst} and \var{skiplast}: the lengths on the original path not to be bent to a cycloid

\var{curvesperhloop}: the number of Bezier curves to approximate a half-loop

\var{sign}: with \code{sign>=0} starts the cycloid to the left of the path, \code{sign<0} to the right.

\var{turnangle}: the angle of perspective on the 3D spring. At
\code{turnangle=0} one sees a sinusoidal curve, at \code{turnangle=90} one
essentially sees a circle.
\end{classdesc}

\begin{classdesc}{smoothed}{radius, softness=1, obeycurv=0, relskipthres=0.01}
This deformer creates a smoothed variant of the original path. The smoothing is
done on the basis of the corners of the original path, not on a global skope!
Therefore, the result might not be what one would draw by hand. At each corner
(or wherever two path elements meet) a piece of length $2\times$\var{radius} is
taken out of the original path and replaced by a curve. This curve is
determined by the tangent directions and the curvatures at its endpoints. Both
are given from the original path, and therefore, the new curve fits into the
gap in a \textit{geometrically smooth} way. Path elements that are shorter than
\var{radius}$\times$\var{relskipthres} are ignored.

The new curve smoothing the corner consists either of one or of two Bezier
curves, depending on the surrounding path elements. If there are straight lines
before and after the new curve, then two Bezier curves are used. This optimises
the bending of curves in rectangular boxes or polygons. Here, the curves have
an additional degree of freedom that can be set with \var{softness} $\in(0,1]$.
If one of the concerned path elements is curved, only one Bezier curve is used
that is (not always uniquely) determined by its geometrical constraints. There
are, nevertheless, some \textit{caveats}:

A curve that strictly obeys the sign and magnitude of the curvature might not
look very smooth in some cases. Especially when connecting a curved with a
straight piece, the smoothed path contains unwanted overshootings. To prevent
this, the parameter default \var{obeycurv=0} releases the curvature constraints a
little: The curvature may then change its sign (still looks smooth for human
eyes) or, in more extreme cases, even its magnitude (does not look so smooth).
If you really need a geometrically smooth path on the basis of Bezier curves,
then set \var{obeycurv=1}.
\end{classdesc}

\begin{classdesc}{parallel}{distance, relerr=0.05, sharpoutercorners=0, dointersection=1,
                       checkdistanceparams=[0.5], lookforcurvatures=11}%
This deformer creates a parallel curve to a given path. The result is similar to
what is usually referred to as the \emph{set with constant distance} to the set of
points on the path. It differs in one important respect, because the
\var{distance} parameter in the deformer is a signed distance. The resulting
parallel normpath is constructed on the level of the original pathitems. For
each of them a parallel pathitem is constructed. Then, they are connected by
circular arcs (or by sharp edges) around the corners of the original path.
Later, everything that is nearer to the original path than distance is cut away.

There are some caveats:
\begin{itemize}
  \item When the original path is too curved then the parallel path would
  contain points with infinte curvature. The resulting path stops at such points
  and leaves the too strongly curved piece out.
  \item When the original path contains self-intersection, then the resulting
  parallel path is not continuous in the parameterisation of the original path.
  It may first take a piece that corresponds to ``later'' parameter values and
  then continue with an ``earlier'' one. Please don't get confused.
\end{itemize}

The parameters are the following:

\var{distance} is the minimal (signed) distance between the original and the
parallel paths.

\var{relerr} is the allowed error in the distance is given by
\code{distance*relerr}.

\var{sharpoutercorners} connects the parallel pathitems by wegde build of
straight lines, instead of taking circular arcs. This preserves the angle of the
original corners.

\var{dointersection} is a boolean for performing the last step, the intersection
step, in the path construction. Setting this to 0 gives the full parallel path,
which can be favourable for self-intersecting paths.

\var{checkdistanceparams} is a list of parameter values in the interval (0,1)
where the distance is checked on each parallel pathitem

\var{lookforcurvatures} is the number of points per normpathitem where its
curvature is checked for critical values

\end{classdesc}

%%% Local Variables:
%%% mode: latex
%%% TeX-master: "manual.tex"
%%% ispell-dictionary: "british"
%%% End:

% \chapter{Module canvas: PostScript interface}

\label{canvas}

\section{Class canvas}

The basic module for the PostScript access in \PyX{} is named
\verb|canvas|. Its main purpose is to provide the class \verb|canvas|,
which presents a collection of visual elements like paths, other
canvases, \TeX{} or \LaTeX{} elements. Eventually, the contents of a canvas 
can be written to a (E)PS file to produce the desired output.

\subsection{Basic usage}

The main \verb|canvas| instance, which we shall by convention always
name \verb|c|, will usually be constructed without giving any
arguments:
\begin{quote}
\begin{verbatim}
import pyx

c = canvas.canvas()
\end{verbatim}
\end{quote}
Basic drawing proceeds then via the construction of a \verb|path|, which 
can subsequently be drawn on the canvas using the method \verb|draw()|:
\begin{quote}
\begin{verbatim}
p = path.line(0, 0, 10, 10)
c.draw(p)
\end{verbatim}
\end{quote}
or more concisely:
\begin{quote}
\begin{verbatim}
c.draw(path.line(0, 0, 10, 10))
\end{verbatim}
\end{quote}
You can modify the appearance of a path by additionally passing 
instances of the class \verb|PathStyle|. For instance, you can draw the 
the above path \verb|p| in blue, as well:
\begin{quote}
\begin{verbatim}
c.draw(p, color.rgb.blue)
\end{verbatim}
\end{quote}
Similarly, it is possible to draw a dashed version of \verb|p|:
\begin{quote}
\begin{verbatim}
c.draw(p, canvas.linestyle.dashed)
\end{verbatim}
\end{quote}
Combining of several \verb|PathStyle|s is of course also possible:
\begin{quote}
\begin{verbatim}
c.draw(p, color.rgb.blue, canvas.linestyle.dashed)
\end{verbatim}
\end{quote}
Furthermore, drawing an arrow at the begin or end of the path is done
in a similar way. You just have to use the provided \verb|barrow| and 
\verb|earrow| instances:
\begin{quote}
\begin{verbatim}
c.draw(p, canvas.barrow.normal, canvas.earrow.large)
\end{verbatim}
\end{quote}

Filling of a path is possible via the \verb|fill| method of the canvas.
Let us for example draw a filled rectangle 
\begin{quote}
\begin{verbatim}
r = path.rect(0, 0, 10, 5)
c.fill(r)
\end{verbatim}
\end{quote}
Alternatively, you can use the class \verb|filled| of the canvas module
in combination with the \verb|draw| method:
\begin{quote}
\begin{verbatim}
c.draw(r, canvas.filled())
\end{verbatim}
\end{quote}

To conclude the section on the drawing of paths, we consider a pretty
sophisicated combination of the above presented \verb|PathStyle|s:
\begin{quote}
\begin{verbatim}
c.draw(p, 
       color.rgb.blue, 
       canvas.earrow.LARge(color.rgb.red,
                           canvas.stroked(canvas.linejoin.round),
                                          canvas.filled(color.rgb.green)))
                                                              
\end{verbatim}
\end{quote}
This draws the path in blue with a pretty large arrow at the end, the outline
of which is red and rounded and which is filled with green.

After you are finished with the composition of your canvas, you can
write it to a file using the method \verb|writetofile()|. It expects the
obligatory argument \verb|filename|, the name of the output
file. To write your results to the file "test.eps" just call it as follows:
\begin{quote}
\begin{verbatim}
c.writetofile("test")
\end{verbatim}
\end{quote}


\subsection{Methods}



The \verb|canvas| class provides the following methods:

\medskip
\begin{tabularx}
  {\linewidth}
  {>{\hsize=.85\hsize}X>{\raggedright\arraybackslash\hsize=1.15\hsize}X}
  \texttt{canvas} method & function \\
  \hline
  \texttt{bbox()} &
  Returns the bounding box enclosing all elements of the canvas.\\
  \texttt{draw(path, *styles)} & 
  Draws the given \texttt{path} on the canvas, \textit{i.e.}\
  \texttt{insert}s it together with the necessary \texttt{newpath},
  \texttt{stroke} sequence, applying the given \texttt{styles}. Styles
  can either be instances of \texttt{canvas.PSAttr} or
  \texttt{canvas.PathDeco}
  (or subclasses thereof).\\
  \texttt{fill(path, *styles)} &
  Fills the given \texttt{path} on the canvas, \textit{i.e.}\
  \texttt{insert}s it together with the necessary \texttt{newpath},
  \texttt{fill} sequence, applying the given \texttt{styles}. Styles can
  either be instances of \texttt{canvas.PSAttr} (or subclasses therof).\\
  \texttt{drawfilled(path, *styles)} &
  Draws and fills the given \texttt{path} on the canvas (\textit{i.e.}\
  \texttt{insert}s it together with the necessary \texttt{newpath},
  \texttt{gsave}, \texttt{stroke}, \texttt{grestore}, \texttt{fill} sequence)
  applying the given \texttt{styles}. Styles can either be instances of
  \texttt{canvas.PSAttr} (or subclasses thereof).\\
  \texttt{insert(PSOps, *styles)} &
  Insert one ore more instances of the class \texttt{base.PSOp} in the
  canvas applying the given \texttt{styles}.  Styles have to be instances
  of the class \texttt{canvas.PSAttr} or derived classes thereof.\\
  \texttt{set(*styles)} &
  Sets the given \texttt{styles} (instances of \texttt{canvas.PSAttr} or
  subclasses) for the rest of the canvas.\\
    \texttt{writetofile(filename, 
      \newline\phantom{writetofile(}paperformat=None, 
      \newline\phantom{writetofile(}rotated=0,
      \newline\phantom{writetofile(}fittosize=0, 
      \newline\phantom{writetofile(}margins="1 t cm")} &
  Writes the canvas to \texttt{filename}. Optionally a
  \texttt{paperformat} can be specified, in which case the output will
  be centered with respect to the corresponding size using the given
  \texttt{margin}. See \texttt{canvas.\_paperformats} for a list of
  known paper formats . Use \texttt{rotated}, if you want to center on
  a $90^\circ$ rotated version of the respective paper format. If
  \texttt{fittosize} is set, the output is additionally scaled to the
  maximal possible size.
\end{tabularx} 
\medskip








%%% Local Variables:
%%% mode: latex
%%% TeX-master: "manual.tex"
%%% End:

% \section{Module \module{document}}
\label{document}

\sectionauthor{J\"org Lehmann}{joergl@users.sourceforge.net} 

\declaremodule{}{document}



The document module contains two classes: \class{document} and
\class{page}. A \class{document} consists of one or several
\class{page}s.


\subsection{Class \class{page}}

A \class{page} is a thin wrapper around a \class{canvas}, which
defines some additional properties of the page.

\begin{classdesc}{page}{canvas, pagename=None,
    paperformat=paperformat.A4, rotated=0, centered=1, fittosize=0,
    margin=1 * unit.t_cm, bboxenlarge=1 * unit.t_pt, bbox=None}
  Construct a new \class{page} from the given \class{canvas} instance.
  A string \var{pagename} and the \var{paperformat} can be
  defined. See below, for a list of known paper formats.
  If \var{rotated} is set, the output is rotated by 90 degrees on the
  page. If \var{centered} is set, the output is centered on the given
  paperformat. If \var{fittosize} is set, the output is scaled to fill
  the full page except for a given \var{margin}. 
  Normally, the bounding box of the canvas is calculated automatically
  from the bounding box of its elements.  Alternatively, you may
  specify the \var{bbox} manually. In any case, the bounding box is
  enlarged on all sides by \var{bboxenlarge}.
\end{classdesc}

\subsection{Class \class{document}}

\begin{classdesc}{document}{pages=[]}
    Construct a \class{document} consisting of a given list of \var{pages}.
\end{classdesc}

A \class{document} can be written to a file using one of the following methods:

\begin{methoddesc}{writeEPSfile}{file, *args, **kwargs}
  Write a single page \class{document} to an EPS file, passing
  \var{args} and \var{kwargs} to the \class{epswriter} instance
  created for writing.
\end{methoddesc}

\begin{methoddesc}{writePSfile}{file, *args, **kwargs}
  Write \class{document} to a PS file, passing \var{args} and
  \var{kwargs} to the \class{pswriter} instance created for writing.
\end{methoddesc}

\begin{methoddesc}{writePDFfile}{file, *args, **kwargs}
  Write \class{document} to a PDF file, passing \var{args} and
  \var{kwargs} to the \class{pdfwriter} instance created for writing.
\end{methoddesc}

\begin{methoddesc}{writetofile}{filename, *args, **kwargs}
  Determine the file type (EPS, PS, or PDF) from the file extension
  of \var{filename} and call the corresponding write method with
  the given arguments \var{arg} and \var{kwargs}.
\end{methoddesc}

\subsection{Class \class{paperformat}}

\begin{classdesc}{paperformat}{width, height, name=None}
Define a \class{paperformat} with the given \var{width} and
\var{height} and the optional \var{name}.
\end{classdesc}

Predefined paperformats are listed in the following table
\medskip
\begin{center}
\begin{tabular}{l|l|l|l}
instance & name & width  & height \\
\hline
\constant{document.paperformat.A0} & A0 & \unit[840]{mm} &
\unit[1188]{mm}\\
\constant{document.paperformat.A0b} &  &\unit[910]{mm} &
\unit[1370]{mm}\\
\constant{document.paperformat.A1} & A1& \unit[594]{mm} &
\unit[840]{mm}\\
\constant{document.paperformat.A2} & A2& \unit[420]{mm} &
\unit[594]{mm}\\
\constant{document.paperformat.A3} & A3 & \unit[297]{mm} & \unit[420]{mm}\\
\constant{document.paperformat.A4} & A4& \unit[210]{mm} & \unit[297]{mm}\\
\constant{document.paperformat.Letter} & Letter & \unit[8.5]{inch} &
\unit[11]{inch}\\
\constant{document.paperformat.Legal} & Legal & \unit[8.5]{inch} & \unit[14]{inch}
\end{tabular}
\end{center}
\medskip



%%% Local Variables:
%%% mode: latex
%%% TeX-master: "manual.tex"
%%% End:

% \chapter{Module text: \TeX/\LaTeX{} interface}
\label{module:text}

\section{Basic functionality}

The \verb|text| module seamlessly integrates the famous typesetting
technique of \TeX/\LaTeX{} into \PyX. The basic procedure is:
\begin{itemize}
\item start \TeX/\LaTeX{} as soon as text creation is requested
\item create boxes containing the requested text on the fly
\item immediately analyse the \TeX/\LaTeX{} output for errors etc.
\item boxes are written into the dvi output
\item box extents are immediately available (they are contained in the
\TeX/\LaTeX{} output)
\item as soon as PostScript needs to be written, stop \TeX/\LaTeX{},
analyse the dvi output and generate the requested PostScript
\item use Type1 fonts for the PostScript generation
\end{itemize}

\section{The texrunner}
Instances of the class \verb|texrunner| represent a \TeX/\LaTeX{}
instance. The keyword arguments of the constructor are listed in the
following table:

\medskip
\begin{tabularx}{\linewidth}{l>{\raggedright\arraybackslash}X}
keyword&description\\
\hline
\texttt{mode}&\texttt{"tex"} (default) or \texttt{"latex"}\\
\texttt{lfs}&Specifies a latex font size file to be used with \TeX. Those files (with the suffix \texttt{.lfs}) can be created by \texttt{createlfs.tex}. Possible values are listed when a requested name could not be found.\\
\texttt{docclass}&\LaTeX{} document class; default is \texttt{"article"}\\
\texttt{docopt}&specifies options for the document class; default is \texttt{None}\\
\texttt{usefiles}$^1$&filenames to be as jobname files for \TeX/\LaTeX{}; default: \texttt{None}; example: \texttt{("spam.aux", "eggs.log")}\\
\texttt{waitfortex}&wait this number of seconds for a \TeX/\LaTeX{} response; default \texttt{5}\\
\texttt{texdebug}&\TeX/\LaTeX{} debug messages (boolean); default \texttt{0}\\
\texttt{dvidebug}&dvi debug messages like \texttt{dvitype} (boolean); default \texttt{0}\\
\texttt{texmessagestart}$^{1,2}$&parsers for the \TeX/\LaTeX{} start message; default: \texttt{texmessage.start}\\
\texttt{texmessagedocclass}$^{1,2}$&parsers for \LaTeX{}s \texttt{\textbackslash{}documentclass} statement; default: \texttt{texmessage.load}\\
\texttt{texmessagebegindoc}$^{1,2}$&parsers for \LaTeX{}s \texttt{\textbackslash{}begin\{document\}} statement; default: \texttt{(texmessage.load, texmessage.noaux)}\\
\texttt{texmessageend}$^{1,2}$&parsers for \TeX{}s \texttt{\textbackslash{}end}/ \LaTeX{}s \texttt{\textbackslash{}end\{document\}} statement; default: \texttt{texmessage.texend}\\
\texttt{texmessagedefaultpreamble}$^{1,2}$&default parsers for preamble statements; default: \texttt{texmessage.load}\\
\texttt{texmessagedefaultrun}$^{1,2}$&default parsers for text statements; default: \texttt{None}\\
\end{tabularx}
\medskip

$^1$
The parameter might contain None, a single entry or a sequence of entries.

$^2$
\TeX/\LaTeX{} message parsers are described in more detail below.

\medskip
The \verb|texrunner| instance provides three methods to be called by
the user. The first method is called \verb|set|. It takes the same
kewword arguments as the constructor and its purpose is to provide an
access to the \verb|texrunner|s settings for a given instance. This is
important for the \verb|defaulttextunner|. The \verb|set| method
fails, when a modification can't be applied anymore (e.g.
\TeX/\LaTeX{} was already started).

Secondly there is a \verb|preamble| method, which can be called before
the \verb|text| method only (see below). It takes a \TeX/\LaTeX{}
expression and optionally one or several \TeX/\LaTeX{} message
parsers. The preamble expressions should be used to perform global
settings, but should not create any \TeX/\LaTeX{} dvi output. In
\LaTeX, the preamble expressions are inserted before the
\verb|\begin{document}| statement.

Last, but first, there is a \verb|text| method. The first two
parameters are the x, y position of the output to be generated. The
third parameter is a \TeX/\LaTeX{} expression and further parameters
are attributes for this command. Those attributes might be
\TeX/\LaTeX{} settings as described below, \TeX/\LaTeX{} message
parsers as described below as well, \PyX{} transformations, and \PyX{}
fill styles (like colors). The \verb|text| method returns a box (see
chapter~\ref{module:box}), which can be inserted into a canvas
instance by its \verb|insert| method to get the text.

\section{\TeX/\LaTeX{} settings}

\begin{description}
\raggedright
\item[Horizontal alignment:] \verb|halign.left| (default),
\verb|halign.center|, \verb|halign.right|, \verb|halign(x)| (\verb|x|
is a value between \verb|0| and \verb|1| standing for left and right,
respectively)
\item[Vertical box:] Usually, \TeX/\LaTeX{} expressions are handled in
horizontal mode (so-called LR-mode in \TeX/\LaTeX; everything goes
into a single line). You may use \verb|parbox(x)|, where \verb|x| is the
width of the text, to switch to a multiline mode (its called vertical
mode in \TeX/\LaTeX).
\begin{figure}
\centerline{\includegraphics{textvalign}}
\caption{valign example}
\label{fig:textvalign}
\end{figure}
\item[Vertical alignment:] \verb|valign.top|, \verb|valign.middle|,
\verb|valign.bottom|; in horizontal mode additionally
\verb|valign.baseline| (default); in vertical mode additionally
\verb|valign.topbaseline| (default), \verb|valign.middlebaseline|, and
\verb|valign.bottombaseline|; see figure~\ref{fig:textvalign} for an
example
\item[Vertical shift:] \verb|vshift.char(lowerratio, heightstr="0")|
(lowers the output by \verb|lowerratio| of the height of
\verb|heightstr|), \verb|vshift.bottomzero=vshift.char(0)| (doesn't
have an effect), \verb|vshift.middlezero=vshift.char(0.5)| (shifts
down by halve of the height of a \verb|0|),
\verb|vshift.topzero=vshift.char(1)| (shifts down by the height of the a
\verb|0|), \verb|vshift.mathaxis| (shifts down by the height of the
mathematical axis)
\item[Mathmode:] \verb|mathmode| switches the mathmode of \TeX/\LaTeX
\item[Font size:] \verb|size.tiny|, \verb|size.scriptsize|,
\verb|size.footnotesize|, \verb|size.small|, \verb|size.normalsize|
(default), \verb|size.large|, \verb|size.Large|, \verb|size.LARGE|,
\verb|size.huge|, \verb|size.Huge|
\end{description}

\section{\TeX/\LaTeX{} message parsers}

Message parsers are used to scan the output of \TeX/\LaTeX. The output
is analysed by a sequence of message parsers. Each of them analyses
the output and remove those parts of the output, it feels responsible
for. If there is nothing left in the end, the message got validated,
otherwise an exception is raised reporting the problem.

\medskip
\begin{tabular}{ll}
parser name&purpose\\
\hline
\texttt{texmessage.load}&loading of files (accept \texttt{(file ...)})\\
\texttt{texmessage.graphicsload}&loading of graphic files (accept \texttt{<file ...>})\\
\texttt{texmessage.ignore}&accept everything as a valid output\\
\end{tabular}
\medskip

More specialised message parsers should become available as required.
Please feal free to contribute (e.g. with ideas/problems; code is
desired as well, of course). There are further message parsers for
\PyX{}s internal use, but we skip them here as they are not
interesting from the users point of view.

\section{The defaulttexrunner instance}
The \verb|defaulttexrunner| is an instance of the class
\verb|texrunner|, which is automatically created by the \verb|text|
module. Additionally, the methods \verb|text|, \verb|preamble|, and
\verb|set| are available as module functions accessing the
\verb|defaulttexrunner|. This single \verb|texrunner| instance is
sufficient in most cases.


% \chapter{Module graph: graph plotting}
\label{graph}
\section{Introductory notes}

The graph module is considered to be in constant, gradual development.
For the moment we concentrate ourself on standard 2d xy-graphs taking
all kind of possible specialties into account like any number of axes.
Architectural decisions play the most substantial role at the moment
and have hopefully already been done that way, that their flexibility
will suffice for future usage in quite different graph applications,
\emph{e.g.} circular 2d graphs or even 3d graphs. We will describe
those parts of the graph module here, which are in a totally usable
state already and are hopefully not to be changed later on. However,
future developments certainly will cause incompatibilities, for
example they are expected to happen for automatic axis ticking (which
will therefore not yet be covered within this manual). At least be
warned: Nobody knows the hole list of things that will break. At the
moment, keeping backwards compatibility in the graph module is not at
all an issue. Although we do not yet claim any backwards compatibility
for the future at all, the graph module is certainly one of the
biggest construction sites within \PyX.

The task of drawing graphs is splitted in quite some subtasks, which
are implemented by classes of its own. We tried to make those
components as independend as it is usefull and possible in order to
make them reuseable for different graph types. They are also
replaceable by the user to get more specialized graph drawing tasks
done without needing to implement a hole graph system. A major
abstraction layer are the so-called graph coordinates. Their range is
generally fixed to $[0;1]$. Only the graph does know about the
conversion between these coordinates and the position at the canvas.
By that, all other components can be reused for different graph
geometries.

\section{Axes}
\label{graph:axes}

A common feature of a graph are axes. An axis is responsible for the
conversion of values to graph coordinates. There are predefined axis
types, namely:
\begin{center}
\begin{tabular}{ll}
axis type&description\\
\hline
\texttt{linaxis}&linear axis\\
\texttt{logaxis}&logarithmic axis\\
\end{tabular}
\end{center}

Additional axis types are likely to be added in the future.

\subsection{Axes properties}

Global properties of an axis are set as named parameters in the axis
constructor. Both predefined axis, the \verb|linaxis| and the
\verb|logaxis|, have the same set of named parameters listed in the
following table:

\medskip
\begin{tabularx}{\linewidth}{l>{\raggedright\arraybackslash}X}
argument name&description\\
\hline
\texttt{title}&axis title\\
\texttt{min}&fixes axis minimum; if not set, it is automatically determined, but this might fail, for example for the $x$-range of functions, when it is not specified there\\
\texttt{max}&as above, but for the maximum\\
\texttt{reverse}&boolean; exchange minimum and maximum (might be used without setting minimum and maximum); if min>max and reverse is set, they cancel each other\\
\texttt{divisor}&numerical divisor for the axis partitioning (its default value is 1)\\
\texttt{suffix}&a suffix to indicate the divisor within an automatic axis labeling\\
\texttt{datavmin}&minimal graph coordinate when adjusting the axis minima to the graph data; default is 0.05\\
\texttt{datavmax}&as above, but for the maximum; default is 0.95\\
\texttt{tickvmin}&minimal graph coordinate for placing ticks to the axis; default is 0\\
\texttt{tickvmax}&as above, but for the maximum; default is 1\\
\texttt{part}&axis partitioning (described below)\\
\texttt{painter}&axis painter (described below)\\
\end{tabularx}
\medskip

\subsection{Partitioning of axes}

The definition of ticks and labels appropriate to an axis range is
called partitioning. The axis partioning within \PyX{} uses rational
arithmetics, which avoids any kind of rounding problems to the cost of
performance. The class \verb|frac| supplies a rational number.
However, a partitioning is composed out of a sorted list of ticks,
where the class \verb|tick| is derived from \verb|frac| and has
additional properties called \verb|ticklevel|, \verb|labellevel|. If
those values are \verb|None| it just means not present, \verb|0| means
tick or label, respectively, \verb|1| means subtick or sublevel and so
on. When \verb|labellevel| is not \verb|None|, a \verb|text| might be
explicitly given, which will get used as the text of that label.

Although there is a rudimentary automatic axis partitioning, the
recommended solution at the moment is a manual axis partitioning,
because the manual axis partitioning will hopefully not break in
future versions, while the automatic axis breaking will change for
sure at least in the results it creates.

There are three different manual partition schemes, a manual
partition, another appropriate for linear axes and a third one for
logarithmic axes.

\subsubsection{Manual partitioning}

The class \verb|manualpart| creates a manual partition as described by
named parameters of the constructor:

\medskip
\begin{tabularx}{\linewidth}{ll>{\raggedright\arraybackslash}X}
argument name&default&description\\
\hline
\texttt{ticks}&\texttt{None}&position of ticks, subticks, etc. (see below)\\
\texttt{labels}&\texttt{None}&position of labels, sublabels, etc. (see below)\\
\texttt{texts}&\texttt{None}&force text at labels, sublabels, etc. (see below)\\
\texttt{mix}&\texttt{()}&ordered tick list to be merged into the result\\
\end{tabularx}
\medskip

The parameters \verb|ticks|, \verb|labels|, and \verb|texts| can
either be a sequence, or a sequence of sequences. (When it is not a
sequence at all, it is converted to a sequence with a single entry.)
When it is a sequence of sequences, than the first sequence stands for
the ticks, labels, and texts of the labels, the second sequence stands
for the subticks, sublabels, and texts of the sublabels, and so on.
When it is just a sequence, it stands for the ticks, labels and texts
of the labels.

The single entries of \verb|ticks| and \verb|labels| can either be a
frac or a string, which will be converted to a frac. However, a float
is not valid in order to avoid a conversion from a float to a frac.
Valid strings are just numbers like \verb|"0.1"|, or fractions like
\verb|"1/10"|.

\subsubsection{Partitioning of linear axes}

The class \verb|linpart| creates a linear partition as described by
named parameters of the constructor:

\medskip
\begin{tabularx}{\linewidth}{ll>{\raggedright\arraybackslash}X}
argument name&default&description\\
\hline
\texttt{ticks}&\texttt{None}&distance between ticks, subticks, etc. (see comment below); when the parameter is \texttt{None}, ticks will get placed at labels\\
\texttt{labels}&\texttt{None}&distance between labels, sublabels, etc. (see comment below); when the parameter is \texttt{None}, labels will get placed at ticks\\
\texttt{extendtick}&\texttt{0}&allow for a range extention to include the next tick of the given level\\
\texttt{extendlabel}&\texttt{None}&as above, but for labels\\
\texttt{epsilon}&\texttt{1e-10}&allow for exceeding the range by that relative value\\
\texttt{texts}&\texttt{None}&as in manualpart\\
\texttt{mix}&\texttt{()}&as in manualpart\\
\end{tabularx}
\medskip

The \verb|ticks| and \verb|labels| can either be a sequence or just a
single entry. When a sequence is provided, the first entry stands for
the tick or label, respectively, the second for the subtick or
sublabel, and so on. The entries can either be a frac or a string,
as in \verb|manualpart|.

\subsubsection{Partitioning of logarithmic axes}

The class \verb|logpart| create a logarithmic partition. The class has
the same arguments as \verb|linpart| upto the interpretation of two
arguments \verb|ticks| and \verb|labels|. Both parameters can contain
just a single entry or a sequence --- the interpretation of those
possibilities is the same as it was for \verb|linpart|. The entries
have to be \verb|shiftfracs|, which contains a \verb|frac| for the
shift, say $s$, and a list of \verb|frac| for the positions, say
$p_i$. Valid positions are then $s^np_i$, where $n$ can be any integer
number. Within \verb|logpart| there are numerous predefined
\verb|shiftfracs|, namely:

\begin{center}
\begin{tabular}{ll}
name&values it descibes\\
\hline
\texttt{shift5fracs1}&1 and multiple of $10^5$\\
\texttt{shift4fracs1}&1 and multiple of $10^4$\\
\texttt{shift3fracs1}&1 and multiple of $10^3$\\
\texttt{shift2fracs1}&1 and multiple of $10^2$\\
\texttt{shiftfracs1}&1 and multiple of $10$\\
\texttt{shiftfracs125}&1, 2, 5 and multiple of $10$\\
\texttt{shiftfracs1to9}&1, 2, \dots, 9 and multiple of $10$\\
\end{tabular}
\end{center}

\subsubsection{Automatic partitioning}

When no explicit axis partitioning is given, an automatic axis
partitioning is already available, but it is still considered to be
under development. A major feature is missing, namely the rating of
possible partitions does not yet attend the label texts, which is
important in order to avoid overlap of label texts. In order to
provide it, the rating of axis partitions has to be moved into the
axis painter. That has just to be done and is considered for the next
major release of \PyX.

\subsection{Painting of axes}

A major task of an axis is the painting of itself. It is done by
instances of \verb|axispainter|, provided to the constructor of an
axis as its painter. The constructor of the axis painter receives a
numerous list of named parameters to modify the axis look. A list of
parameters is provided in the following table:

\medskip
\begin{tabularx}{\linewidth}{l>{\raggedright\arraybackslash}X}
argument name&description\\
\hline
\texttt{innerticklengths}$^{1,4}$&tick length of inner ticks (visual length);\newline default: \texttt{axispainter.defaultticklengths}\\
\texttt{outerticklengths}$^{1,4}$&as before, but for outer ticks; default: \texttt{None}\\
\texttt{tickattrs}$^{2,4}$&stroke attributes for ticks; default: \texttt{()}\\
\texttt{gridattrs}$^{2,4}$&stroke attributes for grid lines; default: \texttt{None}\\
\texttt{zerolineattrs}$^{3,4}$&stroke attributes for a grid line at axis value 0; default: \texttt{()}\\
\texttt{baselineattrs}$^{3,4}$&stroke attributes for the axis baseline;\newline default: \texttt{canvas.linecap.square}\\
\texttt{labeldist}&label distance from axis (visual length); default: \texttt{"0.3 cm"}\\
\texttt{labelattrs}$^{2,4}$&text attributes for labels;\newline default: \texttt{((), tex.fontsize.footnotesize)}\\
\texttt{labeldirection}$^4$&relative label direction (see below); default: \texttt{None}\\
\texttt{labelhequalize}&set width of labels to its maximum (boolean); default: \texttt{0}\\
\texttt{labelvequalize}&set height and depth of labels to their maxima (boolean); default: \texttt{1}\\
\texttt{titledist}&title distance from labels (visual length); default: \texttt{"0.3 cm"}\\
\texttt{titleattrs}$^{3,4}$&text attributes for title; default: \texttt{()}\\
\texttt{titledirection}$^4$&relative title direction (see below);\newline default: \texttt{axispainter.paralleltext}\\
\texttt{titlepos}&title position in graph coordinates; default: \texttt{0.5}\\
\texttt{fractype}&text creation for labels (see below);\newline default: \texttt{axispainter.fractypeauto}\\
\texttt{ratfracsuffixenum}&write suffix at the enumerator (boolean); default: \texttt{1}\\
\texttt{ratfracover}&text for fraction line; default: \texttt{r"\textbackslash over"}\\
\texttt{decfracpoint}&decimal point; default: \texttt{"."}\\
\texttt{expfractimes}&text between factor and decimal power; default: \texttt{r"\textbackslash cdot"}\\
\texttt{expfracpre1}&allow factor 1 before a decimal power (boolean); default: \texttt{0}\\
\texttt{expfracminexp}&minimal exponent for decimal power; default: \texttt{4}\\
\texttt{suffix0}&when a suffix is \texttt{x} write \texttt{0x} instead of \texttt{0} (boolean); default: \texttt{0}\\
\texttt{suffix1}&when a suffix is \texttt{x} write \texttt{1x} instead of \texttt{x} (boolean); default: \texttt{0}\\
\end{tabularx}
\medskip

$^1$
The parameter should be a sequence, where the entries are attributes
for the different levels. When the level is larger then the sequence
length, \verb|None| is assumed. When the parameter is not a sequence,
it is applied to all levels.\\
$^2$
The parameter should be a sequence of sequences, where the entries are
attributes for the different levels. When the level is larger then the
sequence length, \verb|None| is assumed. When the parameter is not a
sequence of sequences, it is applied to all levels.\\
$^3$
The parameter should be a sequence. When the parameter is not a
sequence, the parameter is interpreted as a sequence with a single
entry.\\
$^4$
The feature can be turned off by the value \verb|None|. Within
sequences or sequences of sequences, the value \verb|None| might be
used to turn off the feature for some levels selectively.
\medskip

Relative directions for labels (\verb|labeldirection|) and titles
(\verb|titledirection|) are basically a float number in degree. The
text direction is calculated relatively to the baseline of the axis
and is added as an attribute of the text, when no direction was
already provided. The relative direction prevents upside down text by
flipping it by 180 degrees. For convenience, the two self-explanatory
values \verb|axispainter.paralleltext| and
\verb|axispainter.orthogonaltext| are available.

The \verb|fractype| parameter determines the creation of label texts.
There are three types available, which can be forced by providing them
to the \verb|fractype| parameter. The possibilities are listed in the
following table.

\begin{center}
\begin{tabular}{lll}
\texttt{fractype}&description&example\\
\hline
\texttt{axispainter.fractypedec}&decimal&$0.1$\\
\texttt{axispainter.fractypeexp}&decimal with exponent&$2\cdot 10^4$\\
\texttt{axispainter.fractyperat}&rational&$\displaystyle{{1}\over{2}}$\\
\texttt{axispainter.fractypeauto}&automatic (see below)&\\
\end{tabular}
\end{center}

For the default \verb|axispainter.fractypeauto| the three
possibilities are selected depending on some simple rules:
\verb|axispainter.fractyperat| is used, when the axis provides a
suffix, \verb|axispainter.fractypeexp| is used, when the exponent
exceed \verb|expfracminexp|, and \verb|axispainter.fractypedec| is
used otherwise.

\subsection{Linked axes}

Linked axes can be used whenever an axis should be repeated within a
single graph or even between different graphs although the intrinsic
meaning is to have only one axis plotted several times. The
constructor of \verb|linkaxis| receives the axis it is linked to as
its first parameter. Additionally, the named parameter \verb|title|
contains an axis title (default is \verb|None|) and the named
parameter \verb|painter| refers to an axispainter (default is
\verb|linkaxispainter|). This \verb|linkedaxispainter| is a slightly
modified version of the standard \verb|axispainter|. Hence it can
receive all the parameters as the \verb|axispainter| and only the
default value of the parameter \verb|zerolineattrs| is changed to
\verb|None| compared to the \verb|axispainter| previously discussed.
Additionally, two parameters are added, namely \verb|skipticklevel|
and \verb|skiplabellevel|. They are used to build the tick list to be
plotted at the linked axis. Ticks and labels at levels equal or higher
as the provided values get ignored. The default is \verb|None| (do not
ignore any ticks) for the ticks and \verb|0| (ignore all labels) for
the labels.

\section{Data}
\label{graph:data}

\subsection{List of points}

Instances of the class \verb|data| link a \verb|datafile| and a
\verb|style| (see below; default is \verb|mark|). The link object is
needed in order to be able to plot several data from a singe file
without reading the file several times which would just be a bad
design. However, for easy usage, it is possible to provide a filename
instead of a datafile as the first argument to the constructor of the
class \verb|data| hiding the underlying \verb|datafile| instance
completely from view. This is the preverable solution as long as the
datafile gets used only once.

The additional parameters of the constructor of the class \verb|data|
are named parameters. The values of those parameters describe data
columns which are linked to the names of the parameters within the
style. The data columns can be identified directly via their number or
title, or by means of mathematical expressions, as the following table
will show by some examples.

\begin{center}
\begin{tabular}{ll}
selection method&example\\
\hline
as in \texttt{datafile.getcolumnno}&\texttt{data("test.dat", x=1,}\\
&\texttt{\hphantom{data(}y="result", dy="delta")}\\
by mathematical expressions&\texttt{data("test.dat", x="0.5*\$1",}\\
&\texttt{\hphantom{data(}y="0.5*result", dy="0.5*a", a=3)}\\
\end{tabular}
\end{center}

Note that mathematical expressions get evaluated by
\verb|datafile.addcolumn| and thus the same column identifications
become available.

\subsection{Functions}

The class \verb|function| provides data generation out of a functional
expression. The default style for function plotting is \verb|line|.
The constructor of \verb|function| takes an expression as the first
parameter. The expression must be a string with exactly one equal sign
(\verb|=|). At the left side the result axis identifier must be placed
and at the right side the expression must depend on exactly one
variable axis identifier. Hence, a valid expression looks like
\verb|"y=sin(x)"|. You may use the string format syntax to insert
external parameters, \textit{e.g.} \verb|"y=sin(%f*x)" % a| where
\verb|a| is a float variable.

Additional named parameters of the constructor are:

\medskip
\begin{tabularx}{\linewidth}{ll>{\raggedright\arraybackslash}X}
argument name&default&description\\
\hline
\texttt{min}&\texttt{None}&minimal value for the variable parameter; when \texttt{None}, the axis data range will be used\\
\texttt{max}&\texttt{None}&as above, but for the maximum\\
\texttt{points}&\texttt{100}&number of points to be calculated\\
\texttt{parser}&\texttt{mathtree.parser()}&parser for the mathematical expression\\
\end{tabularx}
\medskip

The expression evaluation takes place at a linear raster of the
variable axis. More advanced methods (detection of rapidely changing
functions, handling of divergencies) are likely to be added in future
releases.

\subsection{Parametric functions}

The class \verb|paramfunction| provides data generation out of a
parametric representation of a function. The default style for
parametric function plotting is \verb|line|. The parameter list of the
constructor of \verb|paramfunction| starts with three parameters
describing the function parameter. The first parameter is a string,
namely the variable name. It is followed by a minimal and maximal
value to be used for that parameter. The next parameter contains an
expression assigning functions to the axis identifiers in a quite
pythonic tuple notation. As an example, such an expression could look
like \verb|"x, y = sin(k), cos(3*k)"|.

Additionally, the two named parameters \verb|points| and \verb|parser|
behave like their equally named counterparts in \verb|function|.

\section{Styles}
\label{graph:styles}

Styles are used to draw data at a graph. A style determines what is
painted and how it is painted. Due to this powerfull approach there
are already some different marker types available and the possibility
to introduce other styles opens even more prospects.

On the other hand there is not yet any support for bar graphs. This is
due to the fact that it might be better implemented together with some
specialized axes. It will be shown in the future, what solution will
arise out of that idea instead of an disposable implementation right
now.

\subsection{Marks}

The class \verb|mark| can be used to plot markers, errorbars and lines
configurable by parameters of the constructor. Providing \verb|None|
to attributes hides the according component.

\medskip
\begin{tabularx}{\linewidth}{ll>{\raggedright\arraybackslash}X}
argument name&default&description\\
\hline
\texttt{mark}&\texttt{changemark.cross()}&marker to be used (see below)\\
\texttt{size}&\texttt{"0.2 cm"}&size of the marker (visual length)\\
\texttt{markattrs}&\texttt{canvas.stroked()}&draw attributes for the marker\\
\texttt{errorscale}&\texttt{0.5}&size of the errorbar caps (relative to the marker size)\\
\texttt{errorbarattrs}&\texttt{()}&stroke attributes for the errorbars\\
\texttt{lineattrs}&\texttt{None}&stroke attributes for the line\\
\end{tabularx}
\medskip

The parameter \verb|mark| has to be a routine, which returns a path to
be drawn (e.g. stoked or filled). There are several those routines
already available in the class \verb|mark|, namely \verb|cross|,
\verb|plus|, \verb|square|, \verb|triangle|, \verb|circle|, and
\verb|diamond|. Furthermore, changeable attributes might be used here
(like the default value \verb|changemark.cross|), see
section~\ref{graph:changeattrs} for details.

The attributes are available as class variables after plotting the
style for outside usage. Additionally, the variable \verb|path|
contains the path of the line (even when it wasn't plotted), which
might be used to get crossing points, fill areas, etc.

Valid data names to be used when providing data to markers are listed
in the following table. The character \verb|X| stands for axis names
like \verb|x|, \verb|x2|, \verb|y|, etc.

\begin{center}
\begin{tabular}{ll}
data name&description\\
\hline
\texttt{X}&position of the marker\\
\texttt{Xmin}&minimum for the errorbar\\
\texttt{Xmax}&maximum for the errorbar\\
\texttt{dX}&relative size of the errorbar: \texttt{Xmin, Xmax = X-dX, X+Xd}\\
\texttt{dXmin}&relative minimum \texttt{Xmin = X-dXmin}\\
\texttt{dXmax}&relative maximum \texttt{Xmax = X+dXmax}\\
\end{tabular}
\end{center}

\subsection{Lines}

The class \verb|line| is inherited from \verb|mark| and is restricted
to line drawing. The constructor takes only \verb|lineattrs| and its
default is set to \verb|changelinestyle()|. The other features of the
mark style are turned off.

\subsection{Rectangles}

The class \verb|rect| draws filled rectangles into a graph. The size
and the position of the rectangles to be plotted can be provided by
the same data names like for the errorbars of the class \verb|mark|.
Indeed, the class \verb|mark| reuses most of the marker code by
inheritance, while modifying the errorbar look into a colored filled
rectangle and turing off the marker itself.

The color to be used for the filling of the rectangles is taken from a
gradient provided to the constructor by the named parameter
\verb|gradient| (default is \verb|color.gradient.Gray|). The data
name \verb|color| is used to select the color out of this gradient.

\subsection{Texts}

Another style to be used within graphs is the class \verb|text|, which
adds the output of text to the class \verb|mark|. The text
position relative to the markers is defined by the two named
parameters \verb|textdx| and \verb|textdy| having a default of
\verb|"0 cm"| and \verb|"0.3 cm"|, respectively, which are by default
interpreted as visual length. A further named parameter
\verb|textattrs| may contain a sequence of text attributes (or just a
single attribute). The default for this parameter is
\verb|tex.halign.center|. Furthermore the constructor of this class
allows all other attributes of the class \verb|mark|.

\subsection{Arrows}

The class \verb|arrow| can be used to plot small arrows into a graph
where the size and direction of the arrows has to be given within the
data. The constructor of the class takes the following parameters:

\medskip
\begin{tabularx}{\linewidth}{ll>{\raggedright\arraybackslash}X}
argument name&default&description\\
\hline
\texttt{linelength}&\texttt{"0.2 cm"}&length of a the arrow line (visual length)\\
\texttt{arrowattrs}&\texttt{()}&stroke attributes\\
\texttt{arrowsize}&\texttt{"0.1 cm"}&size of the arrow (visual length)\\
\texttt{arrowdict}&\texttt{\{\}}&attributes to be used in the \texttt{earrow} constructor\\
\texttt{epsilon}&1e-10&smallest allowed arrow size factor for a arrow to become plotted (avoid numerical instabilities)\\
\end{tabularx}
\medskip

The arrow allows for data names like the mark and introduces
additionally the data names \verb|size| for the arrow size (as an
multiplicator for the sizes provided to the constructor) and
\verb|angle| for the arrow direction (in degree).

\subsection{Iterateable style attributes}
\label{graph:changeattrs}

The attributes provided to the constructors of styles can usually
handle so called iterateable attributes, which are changing itself
when plotting several data sets. Iterateable attributes can be easily
written, but there are already some iterateable attributes available
for the most common cases. For example a color change is done by
instances of the class \verb|colorchange|, where the constructor takes
a gradient. Applying this attribute to a style and using this style at
a sequence of data, the color will get changed lineary along the
gradient from one end to the other. The class \verb|colorchange|
includes inherited classes as class variables, which are called like
the color gradients shown in appendix~\ref{gradientname}. For them the
default gradient is set to the appropriate color gradient.

Another attribute changer is called \verb|changesequence|. The
constructor takes a list of attributes and the attribute changer
cycles through this list whenever a new attribute is requested.
This attribute changer is used to implement the following attribute
changers:

\begin{center}
\begin{tabular}{ll}
attribute changer&description\\
\hline
\texttt{changelinestyle}&iterates linestyles solid, dashed, dotted, dasheddotted\\
\texttt{changestrokedfilled}&iterates \texttt{(canvas.stroked(), canvas.filled())}\\
\texttt{changefilledstroked}&iterates \texttt{(canvas.filled(), canvas.stroked())}\\
\end{tabular}
\end{center}

The class \verb|changemark| can be used to cycle throu markers and it
provides already various specialized classes as class variables. To
loop over all available markers (cross, plus, square, triangle,
circle, and diamond) the equal named class variables can be used. They
start at that marker they are named of. Thus \verb|changemark.cross()|
cycles throu the sequence starting at the cross marker. Furthermore
there are four class variables called \verb|squaretwice|,
\verb|triangletwice|, \verb|circletwice|, and \verb|diamondtwice|.
They cycle throu the four fillable markers, but returning the markers
twice before they go on to the next one. They are intented to be used
in combination with \verb|changestrokedfilled| and
\verb|changefilledstroked|.

\section{Keys}
Sorry, there is not yet any support for graph keys.

\section{X-Y-Graph}

The class \verb|graphxy| draws standard x-y-graphs. It is a subcanvas
and can thus be just inserted into a canvas. The x-axes are named
\verb|x|, \verb|x2|, \verb|x3|, \dots and equally the y-axes. The
number of axes is not limited. All odd numbered axes are plotted at
the bottom (for x axes) and at the left (for y axes) and all even
numbered axes are plotted opposite to them. The lower numbers are
closer to the graph.

The constructor of \verb|graphxy| takes axes as named parameters where
the parameter name is an axis name as just described. Those parameters
refer to an axis instance as they where described in
section~\ref{graph:axes}. When no \verb|x| or \verb|y| is provided,
they are automatically set to instances of \verb|linaxis|. When no
\verb|x2| or \verb|y2| axes are given they are initialized as standard
linkaxis to the axis \verb|x| and \verb|y|. However, you can turn off
the automatism by setting those axes explicitly to \verb|None|.

However, the constructor takes some more attributes, namely first of
all a tex canvas. (This ugly construction is likely to be ommited in
future versions of \PyX{} once a new \TeX{} binding becomes
available.) Other parameters are named and listed in the following
table:

\medskip
\begin{tabularx}{\linewidth}{ll>{\raggedright\arraybackslash}X}
argument name&default&description\\
\hline
\texttt{xpos}&\texttt{"0"}&x position of the graph (user length)\\
\texttt{ypos}&\texttt{"0"}&y position of the graph (user length)\\
\texttt{width}&\texttt{None}&width of the graph area (axes are outside of that range)\\
\texttt{height}&\texttt{None}&as abovem, but for the height\\
\texttt{ratio}&\texttt{goldenrule}&width/height ratio when only a width or height is provided\\
\texttt{backgroundattrs}&\texttt{None}&background attributes for the graph area\\
\texttt{axisdist}&\texttt{"0.8 cm"}&distance between axis (visual length)\\
\end{tabularx}
\medskip

After a graph is constructed, data can be plotted via the \verb|plot|
method. The first argument should be an instance of the data providing
classes described in section~\ref{graph:data}. This first parameter
can also be a list of those instances when you want to iterate the
style you explicitly provide as a second parameter to the plot method.
The plot method returns the style (or a list of styles when a data
list was provided) which was used for plotting. Just as an example you
can thus access the path of a line and fill areas with it and so on.

After the plot method was called once or several times, you should
call the method \verb|finish|. (This is actually needed as long as a
tex canvas gets used for text output and the tex canvas is inserted
into the main canvas before the graph gets inserted.) Finishing a
graph allows for the access to positioning routines which can be quite
usefull to plot additional information into a graph.

Sometimes it is also nice to partly finish a graph. By that you can
even modify the order in which a graph performs its drawing process.
By default the four methods \verb|dolayout|, \verb|dobackground|,
\verb|doaxis|, and \verb|dodata| are called in that order. The method
\verb|dolayout| must always be called first, but this is internally
ensured once you call any of the routines yourself. After
\verb|dolayout| gets called, the method \verb|plot| can not be used
anymore.

To get a position within a graph as a tuple out of some axes values,
the method \verb|pos| can be used. It takes two values for a position
at the x and y axis. By default, the axes named \verb|x| or \verb|y|
are used, but this is changed when the named parameters \verb|xaxis|
and \verb|yaxis| are set to other axes. The graph axes are available
by their name using the dictionary \verb|axes|. Each axis has a method
\verb|gridpath| which is set by the graph. It returns a gridpath for a
given position at the axis.

\section{Examples}

\subsection{A minimal example: plot data from a file}

We plot data from the file \verb|"graph.dat"|:

\begin{quote}
\begin{verbatim}
1   2
2   3
3   8
4  13
5  18
6  21
\end{verbatim}
\end{quote}

The following script creates the file \verb|"graph.eps"|:

\begin{quote}
\begin{verbatim}
from pyx import *

c = canvas.canvas()
t = c.insert(tex.tex())
g = c.insert(graph.graphxy(t, width=10))
g.plot(graph.data("graph.dat", x=1, y=2))
g.finish()
c.writetofile("graph")
\end{verbatim}
\end{quote}

The result looks like:
\begin{quote}
\includegraphics{graph1}
\end{quote}

\subsection{A more advanced function plot}

\begin{quote}
\begin{verbatim}
from pyx import *
from pyx.graph import *

c = canvas.canvas()
t = tex.tex()

a, b = 2, 9

mypainter=axispainter(baselineattrs=canvas.earrow.normal)

g = c.insert(graphxy(t, width=10, x2=None, y2=None,
                     x=linaxis(min=0, max=10,
                               part=manualpart(ticks=(frac(a, 1),
                                                      frac(b, 1)),
                                               texts=("a", "b")),
                               painter=mypainter),
                     y=linaxis(painter=mypainter,
                               part=manualpart())))

line = g.plot(function("y=(x-3)*(x-5)*(x-7)"))
g.finish()

pa = path.path(g.axes["x"].gridpath(a))
pb = path.path(g.axes["x"].gridpath(b))
(splita,), (splitpa,) = line.path.intersect(pa)
(splitb,), (splitpb,) = line.path.intersect(pb)
area = (pa.split(splitpa)[0] <<
        line.path.split(splita, splitb)[1] <<
        pb.split(splitpb)[0].reversed())
area.append(path.closepath())
g.stroke(area, canvas.linewidth.THick,
         canvas.filled(color.gray(0.8)))
t.text(g.pos(0.5*(a+b), 0)[0], 1,
       r"\int_a^b f(x) {\rm d}x", tex.halign.center, tex.style.math)

c.insert(t)
c.writetofile("graph2")
\end{verbatim}
\end{quote}

The result looks like:
\begin{quote}
\includegraphics{graph2}
\end{quote}


% \chapter{Axes\label{axis}}

Axes are a fundamental component of graphs although there might be 
applications outside of the graph system. Internally axes are 
constructed out of components, which handle different tasks axes 
need to fulfill:

\begin{definitions}
\term{axis}
  Basically a container for axis data and the components. It
  implements the conversion of a data value to a graph coordinate of
  range [0:1]. It does also handle the proper usage of the components
  in complicated tasks (\emph{i.e.} combine the partitioner, texter,
  painter and rater to find the best partitioning).
\term{tick}
  Ticks are plotted along the axis. They might be labeled with text as
  well.
\term{partitioner, in the code the short form ``parter'' is used}
  Creates one or several choises of tick lists suitable to a certain
  axis range.
\term{texter}
  Creates labels for ticks when they are not set manually.
\term{painter}
  Responsible for painting the axis.
\term{rater}
  Calculate ratings, which can be used to select the best suitable
  partitioning.
\end{definitions}

The names above map directly to modules which are provided in the
directory \file{graph/axis}. Sometimes it might be convenient to
import the axis directory directly rather than to access iit through 
the graph. This would look like:
\begin{verbatim}
  from pyx import *
  graph.axis.painter() # and the like

  from pyx.graph import axis
  axis.painter() # this is shorter ...
\end{verbatim}

In most cases different implementations are available through
different classes, which can be combined in various ways. There are
various axis examples distributed with \PyX{}, where you can see some
of the features of the axis with a few lines of code each. Hence we
can here directly come to the reference of the available
components.

\section{Axes}

\declaremodule{}{graph.axis.axis}
\modulesynopsis{Axes}

The following classes are part of the module \module{graph.axis.axis}.
However, there is a shortcut to access those classes via
\code{graph.axis} directly.

The position of an axis is defined by an instance of a class providing
the following methods:

\begin{methoddesc}[axispos]{basepath}{x1=None, x2=None}
  Returns a path instance for the base path. \var{x1} and \var{x2}
  define the axis range, the base path should cover.
\end{methoddesc}

\begin{methoddesc}[axispos]{vbasepath}{v1=None, v2=None}
  Like \method{basepath} but in graph coordinates.
\end{methoddesc}

\begin{methoddesc}[axispos]{gridpath}{x}
  Returns a path instance for the grid path at position \var{x}.
  Might return \code{None} when no grid path is available.
\end{methoddesc}

\begin{methoddesc}[axispos]{vgridpath}{v}
  Like \method{gridpath} but in graph coordinates.
\end{methoddesc}

\begin{methoddesc}[axispos]{tickpoint}{x}
  Returns the position of \var{x} as a tuple \samp{(x, y)}.
\end{methoddesc}

\begin{methoddesc}[axispos]{vtickpoint}{v}
  Like \method{tickpoint} but in graph coordinates.
\end{methoddesc}

\begin{methoddesc}[axispos]{tickdirection}{x}
  Returns the direction of a tick at \var{x} as a tuple \samp{(dx, dy)}.
  The tick direction points inside of the graph.
\end{methoddesc}

\begin{methoddesc}[axispos]{vtickdirection}{v}
  Like \method{tickdirection} but in graph coordinates.
\end{methoddesc}

Instances of the following classes can be passed to the \var{**axes}
keyword arguments of a graph. Those instances should only be used once.

\begin{classdesc}{linear}{min=None, max=None, reverse=0, divisor=None, title=None,
                          parter=parter.autolinear(), manualticks=[],
                          density=1, maxworse=2, rater=rater.linear(),
                          texter=texter.mixed(), painter=painter.regular()}
  This class provides a linear axis. \var{min} and \var{max} define the
  axis range. When not set, they are adjusted automatically by the
  data to be plotted in the graph. Note, that some data might want to
  access the range of an axis (\emph{e.g.} the \class{function} class
  when no range was provided there) or you need to specify a range
  when using the axis without plugging it into a graph (\emph{e.g.}
  when drawing an axis along a path).

  \var{reverse} can be set to indicate a reversed axis starting with
  bigger values first. Alternatively you can fix the axis range by
  \var{min} and \var{max} accordingly. When divisor is set, it is
  taken to divide all data range and position informations while
  creating ticks. You can create ticks not taking into account a
  factor by that. \var{title} is the title of the axis.

  \var{parter} is a partitioner instance, which creates suitable ticks
  for the axis range. Those ticks are merged with ticks manually given 
  by \var{manualticks} before proceeding with rating, painting
  \emph{etc.} Manually placed ticks win against those created by the
  partitioner. For automatic partitioners, which are able to calculate
  several possible tick lists for a given axis range, the
  \var{density} is a (linear) factor to favour more or less ticks. It
  should not be stressed to much (its likely, that the result would be
  unappropriate or not at all valid in terms of rating label
  distances). But within a range of say 0.5 to 2 (even bigger for
  large graphs) it can help to get less or more ticks than the default
  would lead to. \var{maxworse} is the number of trials with more
  and less ticks when a better rating was already found. \var{rater}
  is a rater instance, which rates the ticks and the label distances
  for being best suitable. It also takes into account \var{density}.
  The rater is only needed, when the partitioner creates several tick
  lists.

  \var{texter} is a texter instance. It creates labels for those
  ticks, which claim to have a label, but do not have a label string
  set already. Ticks created by partitioners typically receive their
  label strings by texters. The \var{painter} is finally used to
  construct the output. Note, that usually several output
  constructions are needed, since the rater is also used to rate the
  distances between the label for an optimum.
\end{classdesc}

\begin{classdesc}{lin}{...}
  This class is an abbreviation of \class{linear} described above.
\end{classdesc}

\begin{classdesc}{logarithmic}{min=None, max=None, reverse=0, divisor=None, title=None,
                               parter=parter.autologarithmic(), manualticks=[],
                               density=1, maxworse=2, rater=rater.logarithmic(),
                               texter=texter.mixed(), painter=painter.regular()}
  This class provides a logarithmic axis. All parameters work like
  \class{linear}. Only two parameters have a different default:
  \var{parter} and \var{rater}. Furthermore and most importantly, the
  mapping between data and graph coordinates is logarithmic.
\end{classdesc}

\begin{classdesc}{log}{...}
This class is an abbreviation of \class{logarithmic} described above.
\end{classdesc}

\begin{classdesc}{linked}{linkedaxis, painter=painter.linked()}
  This class provides an axis, which is linked to another axis
  instance. This means, it shares all its properties with the axis it
  is linked to except for the painter. Thus a linked axis is painted
  differently.

  A standard application are the \code{x2} and \code{y2} axes in an
  x-y-graph. Linked axes to the \code{x} and \code{y} axes are created
  automatically when not disabled by setting those axes to
  \code{None}. By that, ticks are stroked at both sides of an
  x-y-graph. However, linked axes can be used in other cases as
  well. You can link axes within a graph or between different graphs
  as long as the orgininal axis is finished first (it must fix its
  layout first).
\end{classdesc}

\begin{classdesc}{split}{subaxes, splitlist=[0.5],
                         splitdist=0.1, relsizesplitdist=1,
                         title=None, painter=painter.split()}
  This class provides an axis, splitting the input values to its
  subaxes depending on the range of the subaxes. Thus the subaxes
  need to have fixed range, up to the minimum of the first axis and
  the maximum of the last axis. \var{subaxes} actually takes the list
  of subaxes. \var{splitlist} defines the positions of the splitting
  in graph coordinates. Thus the length of \var{subaxes} must be the
  length of \var{splitlist} plus one. If an entry in \var{splitlist}
  is \code{None}, the axes aside define the split position taking into
  account the ratio of the axes ranges (measured by an internal
  \code{relsize} attribute of each axis).

  \var{splitdist} is the space reserved for a splitting in graph
  coordinates, when the corresponding entry in \var{splitlist} is not
  \code{None}. \var{relsizesplitdist} is the space reserved for the
  splitting in terms, when the corresponding entry in \var{splitlist}
  is \code{None} compared to the \code{relsize} of the axes aside.

  \var{title} is the title of the split axes and \var{painter} is a
  specialized painter, which takes care of marking the axes breaks,
  while the painting of the subaxes are performed by their painters
  themself.
\end{classdesc}

\begin{classdesc}{linkedsplit}{linkedaxis,
                               painter=painter.linkedsplit(),
                               subaxispainter=omitsubaxispainter}
  This class provides an axis, which is linked to an instance of
  \class{split}. The purpose of a linked axis is described in class
  \class{linked} above. \var{painter} replaces the painter from the
  \var{linkedaxis} instance.

  While this class creates linked axes for the subaxes of
  \var{linkedsplit} as well, the question arises what painters to use
  there. When \var{subaxispainter} is not set, no painter is given
  explicitly leaving this decision to the subaxes themself. This will
  lead to omitting all labels and the title. However, you can use a
  changeable attribute of painters in \var{subaxispainter} to replace
  the default.
\end{classdesc}

\begin{classdesc}{bar}{subaxis=None, multisubaxis=None,
                       dist=0.5, firstdist=None, lastdist=None,
                       title=None, painter=painter.bar()}
  This class provides an axis suitable for a bar style. It handles a
  discrete set of values and maps them to distinct ranges in graph
  coordinates. For that, the axis gets a list as data values. The
  first entry is taken to be one of the discrete values valid on this
  axis. All other parameters, lets call them others, are passed to a
  subaxis. When others has only one entry, it is passed as a value,
  otherwise as a list. The result of the conversion done by the
  subaxis is mapped into the graph coordinate range for this discrete
  value. When neither \var{subaxis} nor \var{multisubaxis} is set,
  others must be a single value in the range [0:1]. This value is used 
  for the position at the subaxis without conversion.

  When \var{subaxis} is set, it is used for the conversion of others.
  When \var{multisubaxis} is set, it must be an instance of \var{bar}
  as well. It is then duplicated for each of the discrete values
  allowed for the axis. By that, you can create nested bar axes with
  different discrete values for each discrete value of the axis. It
  is not allowed to set both, \var{subaxis} and \var{multisubaxis}.

  \var{dist} is used as the spacing between the ranges for each
  distinct value. It is measured in the same units as the subaxis
  results, thus the default value of \code{0.5} means half the width
  between the distinct values as the width for each distinct value.
  \var{firstdist} and \var{lastdist} are used before the first and
  after the last value. When set to \code{None}, half of \var{dist}
  is used.

  \var{title} is the title of the split axes and \var{painter} is a
  specialized painter for an bar axis. When \var{multisubaxis} is
  used, their painters are called as well, otherwise they are not
  taken into account.
\end{classdesc}

\begin{funcdesc}{pathaxis}{path, axis, direction=1}
  This function returns a (specialized) canvas containing the axis
  \var{axis} painted along the path \var{path}. \var{direction}
  defines the direction of the ticks. Allowed values are \code{1}
  (left) and \code{-1} (right).
\end{funcdesc}

\section{Ticks}

\declaremodule{}{graph.axis.tick}
\modulesynopsis{Axes ticks}

The following classes are part of the module \module{graph.axis.tick}.

\begin{classdesc}{rational}{x, power=1, floatprecision=10}
  This class implements a rational number with infinite precision. For
  that it stores two integers, the numerator \code{num} and a
  denominator \code{denom}. Note that the implementation of rational
  number arithmetics is not at all complete and designed for its
  special use case of axis parititioning in \PyX{} preventing any
  roundoff errors.

  \var{x} is the value of the rational created by a conversion from
  one of the following input values:
  \begin{itemize}
  \item A float. It is converted to a rational with finite precision
    determined by \var{floatprecision}.
  \item A string, which is parsed to a rational number with full
    precision. It is also allowed to provide a fraction like
    \samp{1/3}.
  \item A sequence of two integers. Those integers are taken as
    numerator and denominator of the rational.
  \item An instance defining instance variables \code{num} and
  \code{denom} like \class{rational} itself.
  \end{itemize}

  \var{power} is an integer to calculate \code{\var{x}**\var{power}}.
  This is useful at certain places in partitioners.
\end{classdesc}

\begin{classdesc}{tick}{x, ticklevel=0, labellevel=0, label=None,
                        labelattrs=[], power=1, floatprecision=10}
  This class implements ticks based on rational numbers. Instances of
  this class can be passed to the \code{manualticks} parameter of a
  regular axis.

  The parameters \var{x}, \var{power}, and \var{floatprecision} share
  its meaning with \class{rational}.

  A tick has a tick level (\emph{i.e.} markers at the axis path) and a
  label lavel (\emph{e.i.} place text at the axis path),
  \var{ticklevel} and \var{labellevel}. These are non-negative
  integers or \var{None}. A value of \code{0} means a regular tick or
  label, \code{1} stands for a subtick or sublabel, \code{2} for
  subsubtick or subsublabel and so on. \code{None} means omitting the
  tick or label. \var{label} is the text of the label. When not set,
  it can be created automatically by a texter. \var{labelattrs} are
  the attributes for the labels.
\end{classdesc}

\section{Partitioners}

\declaremodule{}{graph.axis.parter}
\modulesynopsis{Axes partitioners}

The following classes are part of the module \module{graph.axis.parter}.
Instances of the classes can be passed to the parter keyword argument
of regular axes.

\begin{classdesc}{linear}{tickdist=None, labeldist=None,
                          extendtick=0, extendlabel=None,
                          epsilon=1e-10}
  Instances of this class creates equally spaced tick lists. The
  distances between the ticks, subticks, subsubticks \emph{etc.}
  starting from a tick at zero are given as first, second, third
  \emph{etc.} item of the list \var{tickdist}. For a tick position,
  the lowest level wins, \emph{i.e.} for \code{[2, 1]} even numbers
  will have ticks whereas subticks are placed at odd integer. The
  items of \var{tickdist} might be strings, floats or tuples as
  described for the \var{pos} parameter of class \class{tick}.

  \var{labeldist} works equally for placing labels. When
  \var{labeldist} is kept \code{None}, labels will be placed at each
  tick position, but sublabels \emph{etc.} will not be used. This copy
  behaviour is also available \emph{vice versa} and can be disabled by
  an empty list.

  \var{extendtick} can be set to a tick level for including the next
  tick of that level when the data exceed the range covered by the
  ticks by more then \var{epsilon}. \var{epsilon} is taken relative
  to the axis range. \var{extendtick} is disabled when set to
  \code{None} or for fixed range axes. \var{extendlabel} works similar
  to \var{extendtick} but for labels.
\end{classdesc}

\begin{classdesc}{lin}{...}
This class is an abbreviation of \class{linear} described above.
\end{classdesc}

\begin{classdesc}{autolinear}{variants=defaultvariants,
                              extendtick=0,
                              epsilon=1e-10}
  Instances of this class creates equally spaced tick lists, where the
  distance between the ticks is adjusted to the range of the axis
  automatically. Variants are a list of possible choices for
  \var{tickdist} of \class{linear}. Further variants are build out of
  these by multiplying or dividing all the values by multiples of
  \code{10}. \var{variants} should be ordered that way, that the
  number of ticks for a given range will decrease, hence the distances
  between the ticks should increase within the \var{variants} list.
  \var{extendtick} and \var{epsilon} have the same meaning as in
  \class{linear}.
\end{classdesc}

\begin{memberdesc}{defaultvariants}
  \code{[[tick.rational((1, 1)),
  tick.rational((1, 2))], [tick.rational((2, 1)), tick.rational((1,
  1))], [tick.rational((5, 2)), tick.rational((5, 4))],
  [tick.rational((5, 1)), tick.rational((5, 2))]]}
\end{memberdesc}

\begin{classdesc}{autolin}{...}
This class is an abbreviation of \class{autolinear} described above.
\end{classdesc}

\begin{classdesc}{preexp}{pres, exp}
  This is a storage class defining positions of ticks on a
  logarithmic scale. It contains a list \var{pres} of positions $p_i$
  and \var{exp}, a multiplicator $m$. Valid tick positions are defined
  by $p_im^n$ for any integer $n$.
\end{classdesc}

\begin{classdesc}{logarithmic}{tickpos=None, labelpos=None,
                               extendtick=0, extendlabel=None,
                               epsilon=1e-10}
  Instances of this class creates tick lists suitable to logarithmic
  axes. The positions of the ticks, subticks, subsubticks \emph{etc.}
  are defined by the first, second, third \emph{etc.} item of the list
  \var{tickpos}, which are all \class{preexp} instances.

  \var{labelpos} works equally for placing labels. When \var{labelpos}
  is kept \code{None}, labels will be placed at each tick position,
  but sublabels \emph{etc.} will not be used. This copy behaviour is
  also available \emph{vice versa} and can be disabled by an empty
  list.

  \var{extendtick}, \var{extendlabel} and \var{epsilon} have the same
  meaning as in \class{linear}.
\end{classdesc}

Some \class{preexp} instances for the use in \class{logarithmic} are
available as instance variables (should be used read-only):

\begin{memberdesc}{pre1exp5}
  \code{preexp([tick.rational((1, 1))], 100000)}
\end{memberdesc}

\begin{memberdesc}{pre1exp4}
  \code{preexp([tick.rational((1, 1))], 10000)}
\end{memberdesc}

\begin{memberdesc}{pre1exp3}
  \code{preexp([tick.rational((1, 1))], 1000)}
\end{memberdesc}

\begin{memberdesc}{pre1exp2}
  \code{preexp([tick.rational((1, 1))], 100)}
\end{memberdesc}

\begin{memberdesc}{pre1exp}
  \code{preexp([tick.rational((1, 1))], 10)}
\end{memberdesc}

\begin{memberdesc}{pre125exp}
  \code{preexp([tick.rational((1, 1)), tick.rational((2, 1)), tick.rational((5, 1))], 10)}
\end{memberdesc}

\begin{memberdesc}{pre1to9exp}
  \code{preexp([tick.rational((1, 1)) for x in range(1, 10)], 10)}
\end{memberdesc}

\begin{classdesc}{log}{...}
This class is an abbreviation of \class{logarithmic} described above.
\end{classdesc}

\begin{classdesc}{autologarithmic}{variants=defaultvariants,
                                   extendtick=0, extendlabel=None,
                                   epsilon=1e-10}
  Instances of this class creates tick lists suitable to logarithmic
  axes, where the distance between the ticks is adjusted to the range
  of the axis automatically. Variants are a list of tuples with
  possible choices for \var{tickpos} and \var{labelpos} of
  \class{logarithmic}. \var{variants} should be ordered that way, that
  the number of ticks for a given range will decrease within the
  \var{variants} list.

  \var{extendtick}, \var{extendlabel} and \var{epsilon} have the same
  meaning as in \class{linear}.
\end{classdesc}

\begin{memberdesc}{defaultvariants}
  \code{[([log.pre1exp, log.pre1to9exp], [log.pre1exp,
  log.pre125exp]), ([log.pre1exp, log.pre1to9exp], None),
  ([log.pre1exp2, log.pre1exp], None), ([log.pre1exp3,
  log.pre1exp], None), ([log.pre1exp4, log.pre1exp], None),
  ([log.pre1exp5, log.pre1exp], None)]}
\end{memberdesc}

\begin{classdesc}{autolog}{...}
This class is an abbreviation of \class{autologarithmic} described above.
\end{classdesc}

\section{Texter}

\declaremodule{}{graph.axis.texter}
\modulesynopsis{Axes texters}

The following classes are part of the module \module{graph.axis.texter}.
Instances of the classes can be passed to the texter keyword argument
of regular axes. Texters are used to define the label text for ticks,
which request to have a label, but for which no label text has been specified
so far. A typical case are ticks created by partitioners described
above.

\begin{classdesc}{decimal}{prefix="", infix="", suffix="", equalprecision=0,
                           decimalsep=".", thousandsep="", thousandthpartsep="",
                           plus="", minus="-", period=r"\textbackslash overline\{\%s\}",
                           labelattrs=[text.mathmode]}
  Instances of this class create decimal formatted labels.

  The strings \var{prefix}, \var{infix}, and \var{suffix} are added to
  the label at the beginning, immediately after the plus or minus, and at
  the end, respectively. \var{decimalsep}, \var{thousandsep}, and
  \var{thousandthpartsep} are strings used to separate integer from
  fractional part and three-digit groups in the integer and fractional
  part. The strings \var{plus} and \var{minus} are inserted in front
  of the unsigned value for non-negative and negative numbers,
  respectively.

  The format string \var{period} should generate a period. It must
  contain one string insert operators \samp{\%s} for the period.

  \var{labelattrs} is a list of attributes to be added to the label
  attributes given in the painter. It should be used to setup \TeX{}
  features like \code{text.mathmode}. Text format options like
  \code{text.size} should instead be set at the painter.
\end{classdesc}

\begin{classdesc}{exponential}{plus="", minus="-",
                               mantissaexp=r"\{\{\%s\}\textbackslash cdot10\textasciicircum\{\%s\}\}",
                               skipexp0=r"\{\%s\}",
                               skipexp1=None,
                               nomantissaexp=r"\{10\textasciicircum\{\%s\}\}",
                               minusnomantissaexp=r"\{-10\textasciicircum\{\%s\}\}",
                               mantissamin=tick.rational((1, 1)), mantissamax=tick.rational((10L, 1)),
                               skipmantissa1=0, skipallmantissa1=1,
                               mantissatexter=decimal()}
  Instances of this class create decimal formatted labels with an
  exponential.

  The strings \var{plus} and \var{minus} are inserted in front of the
  unsigned value of the exponent.

  The format string \var{mantissaexp} should generate the exponent. It
  must contain two string insert operators \samp{\%s}, the first for
  the mantissa and the second for the exponent. An alternative to the
  default is \samp{r\textquotedbl\{\{\%s\}\{\e rm e\}\{\%s\}\}\textquotedbl}.

  The format string \var{skipexp0} is used to skip exponent \code{0} and must
  contain one string insert operator \samp{\%s} for the mantissa.
  \code{None} turns off the special handling of exponent \code{0}.
  The format string \var{skipexp1} is similar to \var{skipexp0}, but
  for exponent \code{1}.

  The format string \var{nomantissaexp} is used to skip the mantissa
  \code{1} and must contain one string insert operator \samp{\%s} for
  the exponent. \code{None} turns off the special handling of mantissa
  \code{1}. The format string \var{minusnomantissaexp} is similar
  to \var{nomantissaexp}, but for mantissa \code{-1}.

  The \class{tick.rational} instances \var{mantissamin}\textless
  \var{mantissamax} are minimum (including) and maximum (excluding) of
  the mantissa.

  The boolean \var{skipmantissa1} enables the skipping of any mantissa
  equals \code{1} and \code{-1}, when \var{minusnomantissaexp} is set.
  When the boolean \var{skipallmantissa1} is set, a mantissa equals
  \code{1} is skipped only, when all mantissa values are \code{1}.
  Skipping of a mantissa is stronger than the skipping of an exponent.

  \var{mantissatexter} is a texter instance for the mantissa.
\end{classdesc}

\begin{classdesc}{mixed}{smallestdecimal=tick.rational((1, 1000)),
                         biggestdecimal=tick.rational((9999, 1)),
                         equaldecision=1,
                         decimal=decimal(),
                         exponential=exponential()}
  Instances of this class create decimal formatted labels with an
  exponential, when the unsigned values are small or large compared to
  \var{1}.

  The rational instances \var{smallestdecimal} and
  \var{biggestdecimal} are the smallest and biggest decimal values,
  where the decimal texter should be used. The sign of the value is
  ignored here. For a tick at zero the decimal texter is considered
  best as well. \var{equaldecision} is a boolean to indicate whether
  the decision for the decimal or exponential texter should be done
  globally for all ticks.

  \var{decimal} and \var{exponential} are a decimal and an exponential
  texter instance, respectively.
\end{classdesc}

\begin{classdesc}{rational}{prefix="", infix="", suffix="",
                            numprefix="", numinfix="", numsuffix="",
                            denomprefix="", denominfix="", denomsuffix="",
                            plus="", minus="-", minuspos=0, over=r"{{\%s}\textbackslash over{\%s}}",
                            equaldenom=0, skip1=1, skipnum0=1, skipnum1=1, skipdenom1=1,
                            labelattrs=[text.mathmode]}
  Instances of this class create labels formated as fractions.

  The strings \var{prefix}, \var{infix}, and \var{suffix} are added to
  the label at the beginning, immediately after the plus or minus, and at
  the end, respectively. The strings \var{numprefix},
  \var{numinfix}, and \var{numsuffix} are added to the labels
  numerator accordingly whereas \var{denomprefix}, \var{denominfix},
  and \var{denomsuffix} do the same for the denominator.

  The strings \var{plus} and \var{minus} are inserted in front of the
  unsigned value. The position of the sign is defined by
  \var{minuspos} with values \code{1} (at the numerator), \code{0}
  (in front of the fraction), and \code{-1} (at the denominator).

  The format string \var{over} should generate the fraction. It
  must contain two string insert operators \samp{\%s}, the first for
  the numerator and the second for the denominator. An alternative to
  the default is \samp{\textquotedbl\{\{\%s\}/\{\%s\}\}\textquotedbl}.

  Usually, the numerator and denominator are canceled, while, when
  \var{equaldenom} is set, the least common multiple of all
  denominators is used.

  The boolean \var{skip1} indicates, that only the prefix, plus or minus,
  the infix and the suffix should be printed, when the value is
  \code{1} or \code{-1} and at least one of \var{prefix}, \var{infix}
  and \var{suffix} is present.

  The boolean \var{skipnum0} indicates, that only a \code{0} is
  printed when the numerator is zero.

  \var{skipnum1} is like \var{skip1} but for the numerator.

  \var{skipdenom1} skips the denominator, when it is \code{1} taking
  into account \var{denomprefix}, \var{denominfix}, \var{denomsuffix}
  \var{minuspos} and the sign of the number.

  \var{labelattrs} has the same meaning as for \var{decimal}.
\end{classdesc}

\section{Painter}

\declaremodule{}{graph.axis.painter}
\modulesynopsis{Axes painters}

The following classes are part of the module
\module{graph.axis.painter}. Instances of the painter classes can be
passed to the painter keyword argument of regular axes.

\begin{classdesc}{rotatetext}{direction, epsilon=1e-10}
  This helper class is used in direction arguments of the painters
  below to prevent axis labels and titles being written upside down.
  In those cases the text will be rotated by 180 degrees.
  \var{direction} is an angle to be used relative to the tick
  direction. \var{epsilon} is the value by which 90 degrees can be
  exceeded before an 180 degree rotation is performed.
\end{classdesc}

The following two class variables are initialized for the most common
applications:

\begin{memberdesc}{parallel}
  \code{rotatetext(90)}
\end{memberdesc}

\begin{memberdesc}{orthogonal}
  \code{rotatetext(180)}
\end{memberdesc}

\begin{classdesc}{ticklength}{initial, factor}
  This helper class provides changeable \PyX{} lengths starting from
  an initial value \var{initial} multiplied by \var{factor} again and
  again. The resulting lengths are thus a geometric series.
\end{classdesc}

There are some class variables initialized with suitable values for
tick stroking. They are named \code{ticklength.SHORT},
\code{ticklength.SHORt}, \dots, \code{ticklength.short},
\code{ticklength.normal}, \code{ticklength.long}, \dots,
\code{ticklength.LONG}. \code{ticklength.normal} is initialized with
a length of \code{0.12} and the reciprocal of the golden mean as
\code{factor} whereas the others have a modified initial value
obtained by multiplication with or division by appropriate multiples of 
$\sqrt{2}$.

\begin{classdesc}{regular}{innerticklength=ticklength.normal,
                           outerticklength=None,
                           tickattrs=[],
                           gridattrs=None,
                           basepathattrs=[],
                           labeldist="0.3 cm",
                           labelattrs=[],
                           labeldirection=None,
                           labelhequalize=0,
                           labelvequalize=1,
                           titledist="0.3 cm",
                           titleattrs=[],
                           titledirection=rotatetext.parallel,
                           titlepos=0.5,
                           texrunner=text.defaulttexrunner}
  Instances of this class are painters for regular axes like linear
  and logarithmic axes.

  \var{innerticklength} and \var{outerticklength} are visual \PyX{}
  lengths of the ticks, subticks, subsubticks \emph{etc.} plotted
  along the axis inside and outside of the graph. Provide changeable
  attributes to modify the lengths of ticks compared to subticks
  \emph{etc.} \code{None} turns off the ticks inside and outside the
  graph, respectively.

  \var{tickattrs} and \var{gridattrs} are changeable stroke attributes
  for the ticks and the grid, where \code{None} turns off the feature.
  \var{basepathattrs} are stroke attributes for the axis or
  \code{None} to turn it off. \var{basepathattrs} is merged with
  \samp{[style.linecap.square]}.

  \var{labeldist} is the distance of the labels from the axis base path
  as a visual \PyX{} length. \var{labelattrs} is a list of text
  attributes for the labels. It is merged with
  \samp{[text.halign.center, text.vshift.mathaxis]}.
  \var{labeldirection} is an instance of \var{rotatetext} to rotate
  the labels relative to the axis tick direction or \code{None}.

  The boolean values \var{labelhequalize} and \var{labelvequalize}
  force an equal alignment of all labels for straight vertical and
  horizontal axes, respectively.

  \var{titledist} is the distance of the title from the rest of the
  axis as a visual \PyX{} length. \var{titleattrs} is a list of text
  attributes for the title. It is merged with
  \samp{[text.halign.center, text.vshift.mathaxis]}.
  \var{titledirection} is an instance of \var{rotatetext} to rotate
  the title relative to the axis tick direction or \code{None}.
  \var{titlepos} is the position of the title in graph coordinates.

  \var{texrunner} is the texrunner instance to create axis text like
  the axis title or labels.
\end{classdesc}

\begin{classdesc}{linked}{innerticklength=ticklength.short,
                          outerticklength=None,
                          tickattrs=[],
                          gridattrs=None,
                          basepathattrs=[],
                          labeldist="0.3 cm",
                          labelattrs=None,
                          labeldirection=None,
                          labelhequalize=0,
                          labelvequalize=1,
                          titledist="0.3 cm",
                          titleattrs=None,
                          titledirection=rotatetext.parallel,
                          titlepos=0.5,
                          texrunner=text.defaulttexrunner}
  This class is identical to \class{regular} up to the default values of
  \var{labelattrs} and \var{titleattrs}. By turning off those
  features, this painter is suitable for linked axes.
\end{classdesc}

\begin{classdesc}{split}{breaklinesdist="0.05 cm",
                         breaklineslength="0.5 cm",
                         breaklinesangle=-60,
                         titledist="0.3 cm",
                         titleattrs=None,
                         titledirection=rotatetext.parallel,
                         titlepos=0.5,
                         texrunner=text.defaulttexrunner}
  Instances of this class are suitable painters for split axes.

  \var{breaklinesdist} and \var{breaklineslength} are the distance
  between axes break markers in visual \PyX{} lengths.
  \var{breaklinesangle} is the angle of the axis break marker with
  respect to the base path of the axis. All other parameters have the
  same meaning as in \class{regular}.
\end{classdesc}

\begin{classdesc}{linkedsplit}{breaklinesdist="0.05 cm",
                               breaklineslength="0.5 cm",
                               breaklinesangle=-60,
                               titledist="0.3 cm",
                               titleattrs=None,
                               titledirection=rotatetext.parallel,
                               titlepos=0.5,
                               texrunner=text.defaulttexrunner}
  This class is identical to \class{split} up to the default value of
  \var{titleattrs}. By turning off this feature, this painter is
  suitable for linked split axes.
\end{classdesc}

\begin{classdesc}{bar}{innerticklength=None,
                       outerticklength=None,
                       tickattrs=[],
                       basepathattrs=[],
                       namedist="0.3 cm",
                       nameattrs=[],
                       namedirection=None,
                       namepos=0.5,
                       namehequalize=0,
                       namevequalize=1,
                       titledist="0.3 cm",
                       titleattrs=[],
                       titledirection=rotatetext.parallel,
                       titlepos=0.5,
                       texrunner=text.defaulttexrunner}
  Instances of this class are suitable painters for bar axes.

  \var{innerticklength} and \var{outerticklength} are visual \PyX{}
  lengths to mark the different bar regions along the axis inside and
  outside of the graph. \code{None} turns off the ticks inside and
  outside the graph, respectively. \var{tickattrs} are stroke
  attributes for the ticks or \code{None} to turn all ticks off.

  The parameters with prefix \var{name} are identical to their
  \var{label} counterparts in \class{regular}. All other parameters have
  the same meaning as in \class{regular}.
\end{classdesc}

\begin{classdesc}{linkedbar}{innerticklength=None,
                             outerticklength=None,
                             tickattrs=[],
                             basepathattrs=[],
                             namedist="0.3 cm",
                             nameattrs=None,
                             namedirection=None,
                             namepos=0.5,
                             namehequalize=0,
                             namevequalize=1,
                             titledist="0.3 cm",
                             titleattrs=None,
                             titledirection=rotatetext.parallel,
                             titlepos=0.5,
                             texrunner=text.defaulttexrunner}
  This class is identical to \class{bar} up to the default values of
  \var{nameattrs} and \var{titleattrs}. By turning off those features,
  this painter is suitable for linked bar axes.
\end{classdesc}

\section{Rater}

\declaremodule{}{graph.axis.rater}
\modulesynopsis{Axes raters}

The rating of axes is implemented in \module{graph.axis.rater}. When
an axis partitioning scheme returns several partitioning
possibilities, the partitions need to be rated by a positive number.
The axis partitioning rated lowest is considered best.

The rating consists of two steps. The first takes into account only
the number of ticks, subticks, labels and so on in comparison to
optimal numbers. Additionally, the extension of the axis range by
ticks and labels is taken into account. This rating leads to a
preselection of possible partitions. In the second step, after the
layout of preferred partitionings has been calculated, the distance of 
the labels in a partition is taken into account as well at a smaller
weight factor by default. Thereby partitions with overlapping labels
will be rejected completely. Exceptionally sparse or dense labels will
receive a bad rating as well.

\begin{classdesc}{cube}{opt, left=None, right=None, weight=1}
  Instances of this class provide a number rater. \var{opt} is the
  optimal value. When not provided, \var{left} is set to \code{0} and
  \var{right} is set to \code{3*\var{opt}}. Weight is a multiplicator
  to the result.

  The rater calculates
  \code{\var{width}*((x-\var{opt})/(other-\var{opt}))**3} to rate the
  value \code{x}, where \code{other} is \var{left}
  (\code{x}\textless\var{opt}) or \var{right}
  (\code{x}\textgreater\var{opt}).
\end{classdesc}

\begin{classdesc}{distance}{opt, weight=0.1}
  Instances of this class provide a rater for a list of numbers.
  The purpose is to rate the distance between label boxes. \var{opt}
  is the optimal value.

  The rater calculates the sum of \code{\var{weight}*(\var{opt}/x-1)}
  (\code{x}\textless\var{opt}) or \code{\var{weight}*(x/\var{opt}-1)}
  (\code{x}\textgreater\var{opt}) for all elements \code{x} of the
  list. It returns this value divided by the number of elements in the
  list.
\end{classdesc}

\begin{classdesc}{rater}{ticks, labels, range, distance}
  Instances of this class are raters for axes partitionings.

  \var{ticks} and \var{labels} are both lists of number rater
  instances, where the first items are used for the number of ticks
  and labels, the second items are used for the number of subticks
  (including the ticks) and sublabels (including the labels) and so on
  until the end of the list is reached or no corresponding ticks are
  available.

  \var{range} is a number rater instance which rates the range of the
  ticks relative to the range of the data.

  \var{distance} is an distance rater instance.
\end{classdesc}

\begin{classdesc}{linear}{ticks=[cube(4), cube(10, weight=0.5)],
                          labels=[cube(4)],
                          range=cube(1, weight=2),
                          distance=distance("1 cm")}
  This class is suitable to rate partitionings of linear axes. It is
  equal to \class{rater} but defines predefined values for the
  arguments.
\end{classdesc}

\begin{classdesc}{lin}{...}
  This class is an abbreviation of \class{linear} described above.
\end{classdesc}

\begin{classdesc}{logarithmic}{ticks=[cube(5, right=20), cube(20, right=100, weight=0.5)],
                               labels=[cube(5, right=20), cube(5, right=20, weight=0.5)],
                               range=cube(1, weight=2),
                               distance=distance("1 cm")}
  This class is suitable to rate partitionings of logarithmic axes. It
  is equal to \class{rater} but defines predefined values for the
  arguments.
\end{classdesc}

\begin{classdesc}{log}{...}
  This class is an abbreviation of \class{logarithmic} described above.
\end{classdesc}


% \chapter{Module box: convex box handling}
\label{module:box}

This module has a quite internal character, but might still be useful
from the users point of view. It might also get further enhanced to
cover a broader range of standard arranging problems.

In the context of this module a box is a convex polygon having
optionally a center coordinate, which plays an important role for the
box alignment. The center might not at all be central, but it should
be within the box. The convexity is necessary in order to keep the
problems to be solved by this module quite a bit easier and
unambiguous.

Directions (for the alignment etc.) are usually provided as pairs
(dx, dy) within this module. It is required, that at least one of
these two numbers is unequal to zero. No further assumptions are taken.

\section{Polygon}

A polygon is the most general case of a box. It is an instance of the
class \verb|polygon|. The constructor takes a list of points (which
are (x, y) tuples) in the keyword argument \verb|corners| and
optionally another (x, y) tuple as the keyword argument \verb|center|.
The corners have to be ordered counterclockwise. In the following list
some methods of this \verb|polygon| class are explained:

\begin{description}
\raggedright
\item[\texttt{path(centerradius=None, bezierradius=None,
beziersoftness=1)}:] returns a path of the box; the center might be
marked by a small circle of radius \verb|centerradius|; the corners
might be rounded using the parameters \verb|bezierradius| and
\verb|beziersoftness|. For each corner of the box there may be one value
for beziersoftness and two bezierradii. For convenience, it is not necessary
to specify the whole list (for beziersoftness) and the whole list of
lists (bezierradius) here. You may give a single value and/or a 2-tuple instead.
\item[\texttt{transform(*trafos)}:] performs a list of transformations
to the box
\item[\texttt{reltransform(*trafos)}:] performs a list of
transformations to the box relative to the box center

\begin{figure}
\centerline{\includegraphics{boxalign}}
\caption{circle and line alignment examples (equal direction and
distance)}
\label{fig:boxalign}
\end{figure}

\item[\texttt{circlealignvector(a, dx, dy)}:] returns a vector (a
tuple (x, y)) to align the box at a circle with radius \verb|a| in
the direction (\verb|dx|, \verb|dy|); see figure~\ref{fig:boxalign}
\item[\texttt{linealignvector(a, dx, dy)}:] as above, but align at a
line with distance \verb|a|
\item[\texttt{circlealign(a, dx, dy)}:] as circlealignvector, but
perform the alignment instead of returning the vector
\item[\texttt{linealign(a, dx, dy)}:] as linealignvector, but
perform the alignment instead of returning the vector
\item[\texttt{extent(dx, dy)}:] extent of the box in the direction
(\verb|dx|, \verb|dy|)
\item[\texttt{pointdistance(x, y)}:] distance of the point (\verb|x|,
\verb|y|) to the box; the point must be outside of the box
\item[\texttt{boxdistance(other)}:] distance of the box to the box
\verb|other|; when the boxes are overlapping, \verb|BoxCrossError| is
raised
\item[\texttt{bbox()}:] returns a bounding box instance appropriate to
the box
\end{description}

\section{Functions working on a box list}

\begin{description}
\raggedright
\item[\texttt{circlealignequal(boxes, a, dx, dy)}:] Performs a circle
alignment of the boxes \verb|boxes| using the parameters \verb|a|,
\verb|dx|, and \verb|dy| as in the \verb|circlealign| method. For the
length of the alignment vector its largest value is taken for all
cases.
\item[\texttt{linealignequal(boxes, a, dx, dy)}:] as above, but
performing a line alignment
\item[\texttt{tile(boxes, a, dx, dy)}:] tiles the boxes \verb|boxes|
with a distance \verb|a| between the boxes (in addition the maximal box
extent in the given direction (\verb|dx|, \verb|dy|) is taken into
account)
\end{description}

\section{Rectangular boxes}

For easier creation of rectangular boxes, the module provides the
specialized class \verb|rect|. Its constructor first takes four
parameters, namely the x, y position and the box width and height.
Additionally, for the definition of the position of the center, two
keyword arguments are available. The parameter \verb|relcenter| takes
a tuple containing a relative x, y position of the center (they are
relative to the box extent, thus values between \verb|0| and
\verb|1| should be used). The parameter \verb|abscenter| takes a tuple
containing the x and y position of the center. This values are
measured with respect to the lower left corner of the box. By
default, the center of the rectangular box is set to this lower left
corner.


% \chapter{Module \module{connector}}
\label{connector}

This module provides classes for connecting two \class{box}-instances with
lines, arcs or curves.
All constructors of the following connector-classes take two
\class{box}-instances as first arguments. They return a
\class{normpath}-instance from the first to the second box, starting/ending at
the boxes' outline. The behaviour of the path is determined by the
boxes' center and some angle- and distance-keywords. The resulting \class{connector} will
additionally be shortened by lengths given in the \keyword{boxdists}-keyword (a
list of two lengths, default \code{[0,0]}).

\section{Class \class{line}}

The constructor of the \class{line} class accepts only boxes and the
\keyword{boxdists}-keyword.

\section{Class \class{arc}}

The constructor also takes either the \keyword{relangle}-keyword or a
combination of \keyword{relbulge} and \keyword{absbulge}. The ``bulge'' is the
meant to be a hint of the greatest distance between the connecting arc and the
straight connecting line. (Default: \code{relangle=45},
\code{relbulge=None}, \code{absbulge=None})\medskip

Note that the bulge-keywords override the angle-keyword. When both
\keyword{relbulge} and \keyword{absbulge} are given they will be added.

\section{Class \class{curve}}

The constructor takes both angle- and bulge-keywords. Here, the bulges are
used as distances between bezier-curve control points:\medskip

\keyword{absangle1} or \keyword{relangle1}\\
\keyword{absangle2} or \keyword{relangle2}, where the absolute angle overrides the
relative if both are given. (Default: \code{relangle1=45},
\code{relangle2=45}, \code{absangle1=None}, \code{absangle2=None})\medskip

\keyword{absbulge} and \keyword{relbulge}, where they will be added if both are
given.\\ (Default: \code{absbulge=None}, \code{relbulge=0.39}; these default
values produce output similar to the defaults of \class{arc}.)\medskip

Note that relative angle-keywords are counted in the following way:
\keyword{relangle1} is counted in negative direction, starting at the straight
connector line, and \keyword{relangle2} is counted in positive direction.
Therefore, the outcome with two positive relative angles will always leave the
straight connector at its left and will not cross it.

\section{Class \class{twolines}}

This class returns two connected straight lines. There is a vast variety of
combinations for angle- and length-keywords. The user has to make sure to
provide a non-ambiguous set of keywords:\medskip

\keyword{absangle1} or \keyword{relangle1} for the first angle,\\
\keyword{relangleM} for the middle angle and\\
\keyword{absangle2} or \keyword{relangle2} for the ending angle.
Again, the absolute angle overrides the relative if both are given. (Default:
all five angles are \code{None})\medskip

\keyword{length1} and \keyword{length2} for the lengths of the connecting lines.
(Default: \code{None})


% \chapter{Module epsfile: EPS file inclusion}

With help of the \verb|epsfile.epsfile| class, you can easily embed
another EPS file in your canvas.

\label{epsfile}

%%% Local Variables:
%%% mode: latex
%%% TeX-master: "manual.tex"
%%% End:


% \chapter{Bitmaps}
\section{Introduction}
\PyX{} focuses on the creation of scaleable vector graphics. However,
\PyX{} also allows for the output of bitmap images. Still, the support
for creation and handling of bitmap images is quite limited. On the
other hand the interfaces are build that way, that its trivial to
combine \PyX{} with the ``Python Image Library'', also known as
``PIL''.

The creation of a bitmap can be performed out of some unpacked binary
data by first creating image instances:
\begin{verbatim}
from pyx import *
image_bw = bitmap.image(2, 2, "L", "\0\377\377\0")
image_rgb = bitmap.image(3, 2, "RGB", "\77\77\77\177\177\177\277\277\277"
                                      "\377\0\0\0\377\0\0\0\377")
\end{verbatim}
Now \code{image_bw} is a $2\times2$ grayscale image. The bitmap data
is provided by a string, which contains two black (\code{"\e 0" ==
chr(0)}) and two white (\code{"\e 377" == chr(255)}) pixels. Currently
the values per (colour) channel is fixed to 8 bits. The coloured image
\code{image_rgb} has $3\times2$ pixels containing a row of 3 different
gray values and a row of the three colours red, green, and blue.

The images can then be wrapped into \code{bitmap} instances by:
\begin{verbatim}
bitmap_bw = bitmap.bitmap(0, 1, image_bw, height=0.8)
bitmap_rgb = bitmap.bitmap(0, 0, image_rgb, height=0.8)
\end{verbatim}
When constructing a \code{bitmap} instance you have to specify a
certain position by the first two arguments fixing the bitmaps lower
left corner. Some optional arguments control further properties. Since
in this example there is no information about the dpi-value of the
images, we have to specify at least a \code{width} or a \code{height}
of the bitmap.

The bitmaps are now to be inserted into a canvas:
\begin{verbatim}
c = canvas.canvas()
c.insert(bitmap_bw)
c.insert(bitmap_rgb)
c.writeEPSfile("bitmap")
\end{verbatim}
Figure~\ref{fig:bitmap} shows the resulting output.
\begin{figure}[ht]
\centerline{\includegraphics{bitmap}}
\caption{An introductory bitmap example.}
\label{fig:bitmap}
\end{figure}

\section{Bitmap module}
\declaremodule{}{bitmap}
\modulesynopsis{Bitmap support}

\begin{classdesc}{image}{width, height, mode, data, compressed=None}
  This class is a container for image data. \var{width} and
  \var{height} are the size of the image in pixel. \var{mode} is one
  of \code{\textquotedbl L\textquotedbl}, \code{\textquotedbl
  RGB\textquotedbl} or \code{\textquotedbl CMYK\textquotedbl} for
  grayscale, rgb, or cmyk colours, respectively. \var{data} is the
  bitmap data as a string, where each single character represents a
  colour value with ordinal range \code{0} to \code{255}. Each pixel
  is described by the appropriate number of colour components
  according to \var{mode}. The pixels are listed row by row one after
  the other starting at the upper left corner of the image.

  \var{compressed} might be set to \code{\textquotedbl
  Flate\textquotedbl} or \code{\textquotedbl DCT\textquotedbl} to
  provide already compressed data. Note that those data will be passed
  to PostScript without further checks, \emph{i.e.} this option is for
  experts only.
\end{classdesc}

\begin{classdesc}{jpegimage}{file}
  This class is specialized to read data from a JPEG/JFIF-file.
  \var{file} is either a open file handle (it only has to provide a
  \method{read()} method; the file should be opened in binary mode) or
  a string. In the later case \class{jpegimage} will try to open a
  file named like \var{file} for reading.

  The contents of the file is checked for some JPEG/JFIF format
  markers in order to identify the size and dpi resolution of the
  image for further usage. These checks will typically fail for
  invalid data. The data is not uncompressed, but directly inserted
  into the output stream (for invalid data the result will be invalid
  PostScript). Thus there is no quality loss by recompressing the data
  as it would occur when recompressing the uncompressed stream with
  the lossy jpeg compression method.
\end{classdesc}

\begin{classdesc}{bitmap}{xpos, ypos, image, width=None, height=None,
  ratio=None, storedata=0, maxstrlen=4093, compressmode="Flate",
  flatecompresslevel=6, dctquality=75, dctoptimize=1,
  dctprogression=0}
  \var{xpos} and \var{ypos} are the position of the lower left corner
  of the image. This position might be modified by some additional
  transformations when inserting the bitmap into a canvas. \var{image}
  is an instance of \class{image} or \class{jpegimage} but it can also
  be an image instance from the ``Python Image Library''.

  \var{width}, \var{height}, and \var{ratio} adjust the size of the
  image. At least \var{width} or \var{height} needs to be given, when
  no dpi information is available from \var{image}.

  \var{storedata} is a flag indicating, that the (still compressed)
  image data should be put into the printers memory instead of writing
  it as a stream into the PostScript file. While this feature consumes
  memory of the PostScript interpreter, it allows for multiple usage
  of the image without including the image data several times in the
  PostScript file.

  \var{maxstrlen} defines a maximal string length when \var{storedata}
  is enabled. Since the data must be kept in the PostScript
  interpreters memory, it is stored in strings. While most
  interpreters do not allow for an arbitrary string length (a common
  limit is 65535 characters), a limit for the string length is set.
  When more data needs to be stored, a list of strings will be used.
  Note that lists are also subject to some implemenation limits. Since
  a typical value is 65535 enties, in combination a huge amount of
  memory can be used.

  Valid values for \var{compressmode} currently are
  \code{\textquotedbl Flate\textquotedbl} (zlib compression),
  \code{\textquotedbl DCT\textquotedbl} (jpeg compression), or
  \code{None} (disabling the compression). The zlib compression makes
  use of the zlib module as it is part of the standard Python
  distribution. The jpeg compression is available for those
  \var{image} instances only, which support the creation of a
  jpeg-compressed stream, \emph{e.g.} images from the ``Python Image
  Library'' with jpeg support installed. The compression must be
  disabled when the image data is already compressed.

  \var{flatecompresslevel} is a parameter of the zlib compression.
  \var{dctquality}, \var{dctoptimize}, and \var{dctprogression} are
  parameters of the jpeg compression. Note, that the progression
  feature of the jpeg compression should be turned off in order to
  produce valid PostScript. Also the optimization feature is known to
  produce errors on certain printers.
\end{classdesc}


% \chapter{Module bbox}

\label{bbox}

The \texttt{bbox} module contains the definition of the \texttt{bbox}
class representing bounding boxes of graphical elements like paths,
canvases, etc.\ used in \PyX. Usually, you obtain \texttt{bbox}
instances as return values of the corresponding \texttt{bbox())}
method, but you may also construct a bounding box by yourself.

\section{bbox constructor}

The \texttt{bbox} constructor accepts the following keyword arguments

\medskip
\begin{tabularx}{\linewidth}{l>{\raggedright\arraybackslash}X}
keyword & description\\
\hline
\texttt{llx}&\texttt{None} (default) for $-\infty$ or $x$-position of
the lower left corner of the bbox (in user units)\\
\texttt{lly}&\texttt{None} (default) for $-\infty$ or $y$-position of
the lower left corner of the bbox (in user units)\\
\texttt{urx}&\texttt{None} (default) for $\infty$ or $x$-position of
the upper right corner of the bbox (in user units)\\
\texttt{ury}&\texttt{None} (default) for $\infty$ or $y$-position of
the upper right corner of the bbox (in user units)
\end{tabularx}

\section{bbox methods}

%Instances of the \texttt{bbox} class offer the following methods:
%\medskip

\begin{tabularx}
  {\linewidth}
  {>{\hsize=.85\hsize}X>{\raggedright\arraybackslash\hsize=1.15\hsize}X}
  \texttt{bbox} method & function \\
  \hline
  \texttt{intersects(other)} & returns \texttt{1} if the \texttt{bbox} instance
  and \texttt{other} intersect with each other.\\
  \texttt{transformed(self, trafo)}& returns \texttt{self} transformed
  by transformation \texttt{trafo}.\\
  \texttt{enlarged(all=0, bottom=None,
    \newline\phantom{enlarged(}left=None, top=None,
    \newline\phantom{enlarged(}right=None)} &
  return the bounding box enlarged by the given amount (in visual
  units). \texttt{all} is the default for all other directions, which
  is used whenever \texttt{None} is given for the corresponding
  direction.\\
  \texttt{path()} or \texttt{rect()} & return the \texttt{path} corresponding to the
  bounding box rectangle.\\
  \texttt{height()} & returns the height of the bounding box (in \PyX{}
  lengths).\\
  \texttt{width()} & returns the width of the bounding box (in \PyX{}
  lengths).\\
  \texttt{top()} & returns the $y$-position of the top of the bounding
  box (in \PyX{} lengths).\\
  \texttt{bottom()} & returns the $y$-position of the bottom of the
  bounding box (in \PyX{} lengths).\\
  \texttt{left()} & returns the $x$-position of the left side of the
  bounding box (in \PyX{} lengths).\\
  \texttt{right()} & returns the $x$-position of the right side of the
  bounding box (in \PyX{} lengths).\\
  \end{tabularx}
\medskip

Furthermore, two bounding boxes can be added (giving the bounding box
enclosing both) and multiplied (giving the intersection of both
bounding boxes).

%%% Local Variables:
%%% mode: latex
%%% TeX-master: "manual.tex"
%%% End:

% \chapter{Module color}
\label{color}
\section{Color models}
PostScript provides different color models like grey colors, rgb
colors, hsb colors, and cmyk colors. They are available to \PyX{} by
different color classes, which just pass the colors down to the
PostScript level. This implies, that there are no conversion routines
between different color models available. However, some color model
conversion routines are included in python's standard library in the
module \texttt{colorsym}. Furthermore also the comparision of colors
within a color model is not supported, but might be added in future
versions at least for checking color identity and for ordering grey
colors.

\section{Color classes}
There is a class for each of the supported color models, namely
\verb|grey|, \verb|rgb|, \verb|cmyk|, and \verb|hsb|.

\section{Color instances}
Like all attributes, colors are instances of their specific classes.
The constructors of the color classes take the 

\section{Example}
\begin{quote}
\begin{verbatim}
import pyx

c = canvas.canvas()

c.fill(path.rect(0, 0, 7, 3), color.grey(0.8))
c.fill(path.rect(1, 1, 1, 1), color.rgb.red)
c.fill(path.rect(3, 1, 1, 1), color.rgb.green)
c.fill(path.rect(5, 1, 1, 1), color.rgb.blue)

c.writetofile("color")
\end{verbatim}
\end{quote}

The file \verb|color.eps| is created and looks like:
\begin{quote}
\includegraphics{color}
\end{quote}

% \chapter{Module \module{pattern}}
\label{pattern}

\sectionauthor{J\"org Lehmann}{joergl@users.sourceforge.net} 

This module contains the \class{pattern} class, whichs allows the definition of PostScript Tiling
patterns (cf.\ Sect.~4.9 of the PostScript Language Reference Manual)
which may then be used to fill paths. In addition, a number of
predefined hatch patterns are included.


\declaremodule{}{pattern}

\subsection{Class \class{pattern}}

The classes \class{pattern} and \class{canvas} differ only in their
constructor and in the absence of a \method{writeEPSfile()} method in
the former. The \class{pattern} constructor accepts the following
keyword arguments:

\medskip
\begin{tabularx}{\linewidth}{l>{\raggedright\arraybackslash}X}
keyword&description\\
\hline
\texttt{painttype}&\texttt{1} (default) for coloured patterns or
\texttt{2} for uncoloured patterns\\
\texttt{tilingtype}&\texttt{1} (default) for constant spacing tilings
(patterns are spaced constantly by a multiple of a device pixel),
\texttt{2} for undistorted pattern cell, whereby the spacing may vary
by as much as one device pixel, or \texttt{3} for constant spacing and
faster tiling which behaves as tiling type \texttt{1} but with
additional distortion allowed to permit a more efficient
implementation.\\
\texttt{xstep}&desired horizontal spacing between pattern cells, use
\texttt{None} (default) for automatic calculation from pattern
bounding box.\\
\texttt{ystep}&desired vertical spacing between pattern cells, use
\texttt{None} (default) for automatic calculation from pattern
bounding box.\\
\texttt{bbox}&bounding box of pattern. Use \texttt{None} for an
automatic determination of the bounding box (including an
enlargement by $5$ pts on each side.)\\
\texttt{trafo}&additional transformation applied to pattern or
\texttt{None} (default). This may be used to rotate the pattern or to
shift its phase (by a translation).
\end{tabularx}
\medskip

After you have created a pattern instance, you define the pattern
shape by drawing in it like in an ordinary canvas. To use the pattern,
you simply pass the pattern instance to a \method{stroke()},
\method{fill()}, \method{draw()} or \method{set()} method of the
canvas, just like you would do with a colour, etc.



%%% Local Variables:
%%% mode: latex
%%% TeX-master: "manual.tex"
%%% End:

% \chapter{Module unit}
\label{unit}

\sectionauthor{J\"org Lehmann}{joergl@users.sourceforge.net}

\declaremodule{}{unit}

With the \verb|unit| module \PyX{} makes available classes and
functions for the specification and manipulation of lengths. As usual,
lengths consist of a number together with a measurement unit, e.g.,
\unit[1]{cm}, \unit[50]{points}, \unit[0.42]{inch}.  In addition,
lengths in \PyX{} are composed of the five types ``true'', ``user'',
``visual'', ``width'', and ``\TeX'', e.g., \unit[1]{user cm},
\unit[50]{true points}, $(0.42\ \mathrm{visual} + 0.2\ 
\mathrm{width})$ inch.  As their names indicate, they serve different
purposes. True lengths are not scalable and are mainly used for return
values of \PyX{} functions.  The other length types can be rescaled by
the user and differ with respect to the type of object they are
applied to:

\begin{description}
\item[user length:] used for lengths of graphical objects like
  positions etc.
\item[visual length:] used for sizes of visual elements, like arrows,
  graph symbols, axis ticks, etc.
\item[width length:] used for line widths
\item[\TeX{} length:] used for all \TeX{} and \LaTeX{} output
\end{description}

    When not specified otherwise, all types of lengths are interpreted
in terms of a default unit, which, by default, is \unit[1]{cm}.
You may change this default unit by using the module level function
\begin{funcdesc}{set}{uscale=None, vscale=None, wscale=None,
xscale=None, defaultunit=None}
When \var{uscale}, \var{vscale}, \var{wscale}, or \var{xscale} is not
\keyword{None}, the corresponding scaling factor(s) is redefined to
the given number. When \var{defaultunit} is not \keyword{None}, 
the default unit is set to the given value, which has to be
one of \code{"cm"}, \code{"mm"}, \code{"inch"}, or \code{"pt"}.
\end{funcdesc}

For instance, if you only want thicker lines for a publication
version of your figure, you can just rescale all width lengths using
\begin{verbatim}
unit.set(wscale=2)
\end{verbatim}
Or suppose, you are used to specify length in imperial units. In this,
admittedly rather unfortunate case, just use
\begin{verbatim}
unit.set(defaultunit="inch")
\end{verbatim}
at the beginning of your program.

\section{Class length}

\begin{classdesc}{length}{f, type="u", unit=None}
The constructor of the \class{length} class expects as its first
argument a number \var{f}, which represents the prefactor of the given length.
By default this length is interpreted as a user length (\code{type="u"}) in units
of the current default unit (see \function{set()} function of the \module{unit}
module). Optionally, a different \var{type} may be specified, namely
\code{"u"} for user lengths, \code{"v"} for visual lengths, \code{"w"}
for width lengths, \code{"x"} for \TeX{} length, and \code{"t"} for true
lengths. Furthermore, a different unit may be specified using the \var{unit}
argument. Allowed values are \code{"cm"}, \code{"mm"}, \code{"inch"},
and \code{"pt"}.
\end{classdesc}

Instances of the \class{length} class support addition and substraction either by another \class{length}
or by a number which is then interpeted as being a user length in 
default units, multiplication by a number and division either by another
\class{length} in which case a float is returned or by a number in which
case a \class{length} instance is returned. When two lengths are
compared, they are first converted to meters (using the currently set
scaling), and then the resulting values are compared.

\section{Predefined length instances}

A number of \verb|length| instances are already predefined, which
only differ in there values for \verb|type| and \verb|unit|. They are
summarized in the following table

\medskip
\begin{center}
\begin{tabular}{lll|lll}
name & type & unit & name & type & unit\\
\hline
\constant{m} & user & m & \constant{v\_m} & visual & m\\
\constant{cm} & user & cm & \constant{v\_cm} & visual & cm\\
\constant{mm} & user & mm & \constant{v\_mm} & visual & mm\\
\constant{inch} & user & inch & \constant{v\_inch} & visual & inch\\
\constant{pt} & user & points & \constant{v\_pt} & visual & points\\
\constant{t\_m} & true & m & \constant{w\_m} & width & m\\
\constant{t\_cm} & true & cm & \constant{w\_cm} & width & cm\\
\constant{t\_mm} & true & mm & \constant{w\_mm} & width & mm\\
\constant{t\_inch} & true & inch & \constant{w\_inch} & width & inch\\
\constant{t\_pt} & true & points & \constant{w\_pt} & width & points\\
\constant{u\_m} & user & m & \constant{x\_m} & \TeX & m \\
\constant{u\_cm} & user & cm & \constant{x\_cm} & \TeX & cm \\
\constant{u\_mm} & user & mm & \constant{x\_mm} & \TeX & mm \\
\constant{u\_inch} & user & inch & \constant{x\_inch} & \TeX & inch \\
\constant{u\_pt} & user & points & \constant{x\_pt} & \TeX & points\\

\end{tabular}
\end{center}
\medskip

Thus, in order to specify, e.g., a length of 5 width points, just use
\code{5*unit.w_pt}.

\section{Conversion functions}
If you want to know the value of a \PyX{} length in certain units, you
may use the predefined conversion functions which are given in the
following table
\begin{center}
\begin{tabular}{ll}
function & result \\
\hline
\texttt{tom(l)} & \texttt{l} in units of m\\
\texttt{tocm(l)} & \texttt{l} in units of cm\\
\texttt{tomm(l)} & \texttt{l} in units of mm\\
\texttt{toinch(l)} & \texttt{l} in units of inch\\
\texttt{topt(l)} & \texttt{l} in units of points\\
\end{tabular}
\end{center}
If \verb|l| is not yet a \verb|length| instance but a number, it first
is interpreted as a user length in the default units. 



%%% Local Variables:
%%% mode: latex
%%% TeX-master: "manual.tex"
%%% End:

% \chapter{Module trafo: linear transformations}

\label{trafo}

With the  \verb|trafo| module \PyX\ supports linear transformations, which can 
then be applied to canvases,  B\'ezier paths and other objects. It consists
of the main class \verb|trafo| representing a general linear
transformation and subclasses thereof, which provide special operations
like translation, rotation, scaling, and mirroring.

\section{Class trafo}

The \verb|trafo| class represents a general linear
transformation, which is defined for a vector $\vec{x}$ as
\begin{displaymath}
  \vec{x}' = \mathsf{A}\, \vec{x} + \vec{b}\ ,
\end{displaymath}
where $\mathsf{A}$ is the transformation matrix and $\vec{b}$ the
translation vector. The transformation matrix must not be singular,
\textit{i.e.} we require $\det \mathsf{A} \ne 0$.



Multiple \verb|trafo| instances can be multiplied, corresponding to a
consecutive application of the respective transformation. Note that
\verb|trafo1*trafo2| means that \verb|trafo1| is applied after
\verb|trafo2|, \textit{i.e.} the new transformation is given 
by $\mathsf{A} = \mathsf{A}_1 \mathsf{A}_2$ and
$\vec{b} = \mathsf{A}_1 \vec{b}_2 + \vec{b}_1$.  Use the \verb|trafo|
methods described below, if you prefer thinking the other way round.
The inverse of a transformation can be obtained via the \verb|trafo|
method \verb|inverse()|, defined by the inverse $\mathsf{A}^{-1}$ of
the transformation matrix and the translation vector
$-\mathsf{A}^{-1}\vec{b}$.

The methods of the \verb|trafo| class are summarized in the following
table.

\medskip
\begin{tableii}{l|l}{textrm}{\texttt{trafo} method}{function}
\lineii{\texttt{\_\_init\_\_(matrix=((1,0),(0,1)), vector=(0,0)):}}{create new \texttt{trafo} instance with transformation \texttt{matrix} and \texttt{vector}.}
\lineii{\texttt{apply(x, y)}}{apply \texttt{trafo} to point vector $(\mathtt{x}, \mathtt{y})$.}
\lineii{\texttt{inverse()}}{returns inverse transformation of \texttt{trafo}.}
\lineii{\texttt{mirrored(angle)}}{returns \texttt{trafo} followed by mirroring at line through $(0,0)$ with  direction \texttt{angle} in degrees.}
\lineii{\texttt{rotated(angle, x=None, y=None)}}{returns \texttt{trafo} followed by rotation by \texttt{angle} degrees around point $(\mathtt{x}, \mathtt{y})$, or $(0,0)$, if not given.}
\lineii{\texttt{scaled(sx, sy=None, x=None, y=None)}}{returns \texttt{trafo} followed by scaling with scaling factor \texttt{sx} in $x$-direction, \texttt{sy} in $y$-direction ($\mathtt{sy}=\mathtt{sx}$, if not given) with scaling center $(\mathtt{x}, \mathtt{y})$, or $(0,0)$, if not given.}
\lineii{\texttt{translated(x, y)}}{returns \texttt{trafo} followed by translation by vector $(\mathtt{x}, \mathtt{y})$.}
\lineii{\texttt{slanted(a, angle=0, x=None, y=None)}}{returns \texttt{trafo} followed by XXX}
\end{tableii}
\medskip



\section{Subclasses of trafo}

The \verb|trafo| module provides a number of subclasses of
the \verb|trafo| class, each of which corresponds to one \verb|trafo|
method. They are listed in the following table:

\medskip
\begin{tableii}{l|l}{textrm}{\texttt{trafo} subclass}{function}
\lineii{\texttt{mirror(angle)}}{mirroring at line through $(0,0)$ with direction  \texttt{angle} in degrees.}
\lineii{\texttt{rotate(angle, x=None, y=None)}}{rotation by \texttt{angle} degrees around point $(\mathtt{x}, \mathtt{y})$, or $(0,0)$, if not given.}
\lineii{\texttt{scale(sx, sy=None, x=None, y=None)}}{scaling with scaling factor \texttt{sx} in $x$-direction, \texttt{sy} in $y$-direction ($\mathtt{sy}=\mathtt{sx}$, if not given) with scaling center $(\mathtt{x}, \mathtt{y})$, or $(0,0)$, if not given.}
\lineii{\texttt{translate(x, y)}}{translation by vector $(\mathtt{x}, \mathtt{y})$.}
\lineii{\texttt{slant(a, angle=0, x=None, y=None)}}{XXX}
\end{tableii}
\medskip


% \section{Examples}



%%% Local Variables:
%%% mode: latex
%%% TeX-master: "manual.tex"
%%% End:

% \appendix
% \chapter{Named colors}
\centerline{\includegraphics{colorname}}

% \chapter{Named palettes}
\label{palettename}
\centerline{\includegraphics{palettename}}

% \chapter{Path styles and arrows in canvas module}
\label{pathstyles}
\centerline{\includegraphics{pathstyles}}

% \chapter{Arrows in deco module}
\label{arrows}
\centerline{\includegraphics{arrows}}


\end{document}
