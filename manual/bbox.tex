\chapter{Module bbox}

\label{bbox}

The \texttt{bbox} module contains the definition of the \texttt{bbox}
class representing bounding boxes of graphical elements like paths,
canvases, etc.\ used in \PyX. Usually, you obtain \texttt{bbox}
instances as return values of the corresponding \texttt{bbox())}
method, but you may also construct a bounding box by yourself.

\section{bbox constructor}

The \texttt{bbox} constructor accepts the following keyword arguments

\begin{tableii}{l|l}{textrm}{keyword}{description}
\lineii{\texttt{llx}}{\texttt{None} (default) for $-\infty$ or $x$-position of the lower left corner of the bbox (in user units)}
\lineii{\texttt{lly}}{\texttt{None} (default) for $-\infty$ or $y$-position of the lower left corner of the bbox (in user units)}
\lineii{\texttt{urx}}{\texttt{None} (default) for $\infty$ or $x$-position of the upper right corner of the bbox (in user units)}
\lineii{\texttt{ury}}{\texttt{None} (default) for $\infty$ or $y$-position of the upper right corner of the bbox (in user units)}
\end{tableii}

\section{bbox methods}

\begin{tableii}{l|l}{textrm}{\texttt{bbox} method}{function}
\lineii{\texttt{intersects(other)}}{returns \texttt{1} if the \texttt{bbox} instance and \texttt{other} intersect with each other.}
\lineii{\texttt{transformed(self, trafo)}}{returns \texttt{self} transformed by transformation \texttt{trafo}.}
\lineii{\texttt{enlarged(all=0, bottom=None, left=None, top=None, right=None)}}{return the bounding box enlarged by the given amount (in visual units). \texttt{all} is the default for all other directions, which is used whenever \texttt{None} is given for the corresponding direction.}
\lineii{\texttt{path()} or \texttt{rect()}}{return the \texttt{path} corresponding to the bounding box rectangle.}
\lineii{\texttt{height()}}{returns the height of the bounding box (in \PyX{} lengths).}
\lineii{\texttt{width()}}{returns the width of the bounding box (in \PyX{} lengths).}
\lineii{\texttt{top()}}{returns the $y$-position of the top of the bounding box (in \PyX{} lengths).}
\lineii{\texttt{bottom()}}{returns the $y$-position of the bottom of the bounding box (in \PyX{} lengths).}
\lineii{\texttt{left()}}{returns the $x$-position of the left side of the bounding box (in \PyX{} lengths).}
\lineii{\texttt{right()}}{returns the $x$-position of the right side of the bounding box (in \PyX{} lengths).}
\end{tableii}

Furthermore, two bounding boxes can be added (giving the bounding box
enclosing both) and multiplied (giving the intersection of both
bounding boxes).

%%% Local Variables:
%%% mode: latex
%%% TeX-master: "manual.tex"
%%% End:
