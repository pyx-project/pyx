\chapter{Module graph: graph plotting}
\label{graph}
\section{Introductory notes}
The graph module is considered to be in constant, gradual development.
For the moment we concentrate ourself on standard 2d xy-graphs taking
all kind of possible specialties into account like any number of axes.
Architectural decisions play the most substantial role at the moment
and have hopefully already been done that way, that their flexibility
will suffice for future usage in quite different graph applications,
\emph{e.g.} circular 2d graphs or even 3d graphs. We will describe
those parts of the graph module here, which are in a totally usable
state already and are hopefully not to be changed later on. However,
future developments certainly will cause incompatibilities, for
example they are expected to happen for automatic axis ticking (which
will therefore not yet be covered in this manual) or just by splitting
the graph module into several parts. But at least be warned: Nobody
knows the hole list of things that will break. At the moment, keeping
backwards compatibility in the graph module is not at all an issue.
Although we do not yet claim any backwards compatibility for the
future at all, the graph module is certainly one of the biggest
construction site within \PyX.

The task of drawing graphs is splitted in quite some subtasks, which
are implemented by classes of its own. We've tried hard to make those
parts as independend as it is usefull and possible in order to make
them reuseable for other graph types and also to make the replaceable
by the user to get more specialized graph drawing tasks without
needing to reimplement a hole graph system again.

We should mention right here, that a graph introduces own coordinates,
called graph coordinates. The range of graph coordinates is $[0;1]$.
Only the graph does know about the conversion between these
coordinates and the actual position on the postscript canvas.

\section{Axes}

A major feature of a graph are axes. An axis is first of all
responsible for the conversion of values to graph coordinates. This
means, if you have an axis for the range $[10;20]$ than it converts it
to the range $[0;1]$. This conversion might be lineary, but it could
also be logarithmic or whatever else.

Secondly the axis manages its range, e.g. the minimum and the maximum
value. It might be set right at the beginning of the axis creation
(and thus staying fixed all the time) or it will be set appropriate to
the data ranges it recieves.

Finally, a major task of an axis is the painting of itself. For that
it has to calculate positions of ticks and labels and draw them.

\subsection{Axis parting}

\subsection{Axis painting}

\section{Data}

\subsection{List of points}

\subsection{Functions}

\subsection{Parametric functions}

\section{Styles}

\subsection{Lines}

\subsection{Marks}

\subsection{Own styles}

\section{Keys}
Sorry, there is not yet any support for graph keys.

\section{X-Y-Graph}

